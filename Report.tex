\documentclass[a4paper,10pt]{article}

\usepackage{amsmath}
\usepackage{amssymb}
\usepackage[dvipsnames]{xcolor}

\makeatletter
\@ifpackageloaded{tex4ht}{%
  \usepackage[dvips]{graphicx}
}{%
  \usepackage[pdftex]{graphicx}
}
\makeatother

\usepackage{listings}
\usepackage{lmodern}
\usepackage{mdframed}
\usepackage{url}

\usepackage{keyval}

\usepackage[plain]{fancyref}
\usepackage[breakable,minted,skins]{tcolorbox}

\usepackage{geometry}

\DeclareGraphicsRule{*}{mps}{*}{}

\newtcbinputlisting[auto counter]{\libraryCodeFromFile}[3][]{%
  listing engine=minted,
  minted language=rust,
  minted options={linenos,breaklines,fontsize=\footnotesize,tabsize=2,firstnumber=1},
  listing file={Code/Rust/#2},
  listing only,
  size=title,
  breakable,
  boxrule=0.5mm,
  colback=white,
  coltext=Black!95!,
  title=\small{\thetcbcounter}: \UScore{#2},
  label=lst:#3
}

\newtcbinputlisting{\clientCodeFromFile}[2][firstline=1,lastline=-1]{%
  listing engine=minted,
  minted language=rust,
  minted options={autogobble,breaklines,escapeinside=||,fontsize=\footnotesize,tabsize=2,firstnumber=1,#1},
  listing file={Code/Rust/#2},
  listing only,
  breakable,
  frame hidden,
  arc=0mm,
  left=5mm,
  colframe=white,
  boxrule=0mm,
  colback=white,
  coltext=black
}

\newtcbinputlisting[use counter from=libraryCodeFromFile]{\exampleCodeFromFile}[4][firstline=1,lastline=-1]{%
  listing engine=minted,
  minted language=rust,
  minted options={autogobble,linenos,breaklines,escapeinside=||,fontsize=\footnotesize,tabsize=2,firstnumber=1,#1},
  listing file={Code/Rust/#2},
  listing only,
  size=title,
  breakable,
  boxrule=0.5mm,
  colback=white,
  coltext=Black!95!,
  title=\small{Listing \thetcbcounter}: #4,
  label=lst:#3
}

\newtcbox{\highlight}[1][red]{%
    on line,
    arc=0pt,
    outer arc=0pt,
    colback=#1!40!white,
    colframe=#1!50!white,
    boxsep=0pt,
    left=1pt,
    right=1pt,
    top=2pt,
    bottom=2pt,
    boxrule=0pt,
    bottomrule=1pt,
    toprule=1pt
}

\DeclareUrlCommand\UScore{\urlstyle{rm}}

\newcommand*{\fancyreflstlabelprefix}{lst}
\frefformat{plain}{\fancyreflstlabelprefix}{listing\fancyrefdefaultspacing#1}
\Frefformat{plain}{\fancyreflstlabelprefix}{Listing\fancyrefdefaultspacing#1}

\newmdenv[linecolor=BlueGreen,frametitle=Note,frametitlebackgroundcolor=BlueGreen!40!]{notebox}

\title
{
   \includegraphics[width=12cm]{up_logo.png} \\
   \vspace{2cm}
   \textbf{COS700 Research} \\ \vspace{0.5cm}
   \textbf{Design Pattern Metaprogramming Foundations in Rust\\ \large A Study of Abstract Factory and Visitor} \\ \vspace{0.5cm}
   \textbf{Student number:} u19239395 \\ \vspace{0.5cm}
   \textbf{Supervisor}: \\ Dr. Linda Marshall
}

\date{6 November 2020}
\begin{document}
\author{}

\maketitle
\pagenumbering{gobble}

\newpage
\linespread{1.25}

\section*{Abstract}
Design patterns are generic solutions to common software design problems.
This report focuses on a foundation for Rust metaprogramming macros to automatically create implementations for design patterns at compile-time.
The foundation is derived by creating a macro to implement the Abstract Factory design pattern.
Thereafter, the foundation is applied to a macro implementing the Visitor design pattern.

Both macro implementations need to know what a manual implementation for each design pattern will look like in Rust.
Thus, a manual implementation for each design pattern is also presented after exploring the Rust language.
The macro implementations will create code that is identical to the manual implementations.
The metaprogramming style used by Rust is also explored in this report to understand how Rust metaprogramming compares to metaprogramming in other languages.

\section*{Keywords:}
Metaprogramming,
Rust,
Design Patterns,
Procedural Macros,
Abstract Factory,
Visitor

\newpage
\pagenumbering{arabic}

\section{Introduction}
This report will focus on the possibility and foundations needed to implement design patterns using metaprogramming in the Rust language.
Software design is constantly faced with solving the same problems.
The design solutions to these problems became known as design patterns.
However, manually programming a design pattern is a tedious task that can be solved using metaprogramming.

Rust is a new programming language created by Mozilla and was announced the most loved language for the 5th consecutive year, according to the 2020 Developer Survey by StackOverflow\footnote{https://insights.stackoverflow.com/survey/2020\#technology-most-loved-dreaded-and-wanted-languages}.
Rust also has metaprogramming support.
Thus, this report's first objective is to see if Rust's metaprogramming has the capacity to implement design patterns.
If Rust's metaprogramming is capable enough, this report will propose a foundation for implementing any design pattern using Rust's metaprogramming.

In \Fref{sec:design-patterns}, this report will explore the need for design patterns and how they came about.
Next, the definition and design of the Abstract Factory pattern and the Visitor pattern will be created.
Both designs will be implemented in Rust in \Fref{sec:reporting}.

\Fref{sec:metaprogramming} will identify different types of metaprogramming and classify the types Rust uses.
Some lesser-used programming concepts present in Rust will be explored in \Fref{sec:rust}.

Finally, \Fref{sec:reporting} will create implementations for the Abstract Factory and Visitor pattern in the same way a programmer will do manually.
The section will then create a Rust macro to write the same Abstract Factory implementation as the manual implementation.
Using the foundations for the Abstract Factory implementation, the section will create a Rust macro for the Visitor implementation.
\section{Design Patterns}
Software design focuses on designing and implementing software to solve a particular problem \cite{ieee_1016-2009, satzinger_15_01}.
There should be no surprise to see some problems repeating themselves with time \cite{keshvari_11_01}.
The solutions to these problems are the same each time.
But a novice designer facing any of these repeated problems for the first time will try to solve them from first principles \cite{gamma_94_01, sonnentag_98_01}.
When the solution proves flawed or misunderstood some weeks later, a small improvement will be made to the solution \cite{zhu_05_01, ieee_1016-2009}.
These improvements are repeated until all the flaws are removed from the solution \cite{stephens_15_01, satzinger_15_01}.

On the other hand, seasoned designers create good designs from their own or their colleagues' past experiences \cite{sonnentag_98_01}.
These designs focus not only on immediate development but also on the development needed during maintenance \cite{kerievsky_05_01,gamma_94_01}.
These solutions are easy to find in mature libraries and projects \cite{gamma_94_01}.
Unfortunately, novices are unlikely to get exposure to these projects \cite{zhu_05_01} or are just overwhelmed by their size \cite{hu_18_01}.
Having exposure to these projects will allow novices to jump to the good design directly.
This saves time on the iteration process \cite{satzinger_15_01}.

But rather than taking novices to the projects, it might be possible to take the designs to the novices \cite{kim_03_01}.
This is exactly what happened in the 90s.
Gamma et al., which the rest of this report will refer to as the Gang of Four (GoF), took some of the repeated designs in projects and documented them in ``Design Patterns: Elements of Reusable Object-Oriented Software'' \cite{gamma_94_01}.

Each pattern is documented with a name, the problem it is solving, the solution, and the consequences of using it.
Thus, each pattern is an explicit specification for the solution's design while the name becomes a vocabulary encapsulating the specification \cite{gamma_94_01, bulajic_12_01}.
The GoF also groups the patterns into 3 categories: creational, structural, and behavioral.
This report will focus on one creational pattern - \textit{Abstract Factory} - and one behavioral pattern - \textit{Visitor}.
No Structural patterns will be discussed.
Since the \textit{Factory Method} pattern can be used to implement the \textit{Abstract Factory} pattern \cite{nykonenko_12_01, gamma_94_01}, this section will cover \textit{Factory Method} too.
A discussion of \textit{Factory Method}, \textit{Abstract Factory}, and \textit{Visitor} follows.

\subsection{Factory Method}
When an object is created, an isolated function/method may not care which concrete version gets created.
It may only care about an abstract definition of the object to perform its duties.
The Factory Method design pattern proposes to solve three variants of this problem. \cite{gamma_94_01}

\subsubsection{Problems}
The first problem is when a function does not have enough information to determine the concrete object it should create.
The method/function only knows the abstract object it wants.
An example of this is the button on a confirmation dialog.
The abstract confirmation process only needs to create a button.
Whether this is a blue button used during saving or a glossy button used when installing is not the abstract confirmation process's concern.

Problem two follows on from problem one.
Needing to create more than one button means logic to decide on a button to create.
Duplicating this logic at each instantiation will complicate maintenance.
Assume that the blue button is decommissioned as part of a new facelift in favor of the red button.
Updating every line instantiating a blue button will take maintenance time and is error-prone.

Having a superclass delegate the creation responsibility to a subclass is a third problem.
Since all Graphical User Interface (GUI) dialogs follow the same process, the design may call for abstracting the common code into an abstract class.
The abstract class will create the dialog, draw the needed elements on it, and destroy it once done.
However, the concrete open dialog and concrete save dialog will need different buttons.
The abstract class will use virtual methods on the concretes classes to instantiate the correct button for drawing.

\subsubsection{Solution}
The solution will focus on the first two problems since they relate to Abstract Factory.
The first problem calls for an abstraction of the product being created.
This will allow the confirmation process to function against an abstract button and not a blue or glossy one.

Problem two calls for the isolation of a button's instantiation from the decision logic.
This means another abstract class - called \textit{Factory} - to hold the instantiation of a concrete product.
The logic will decide which concrete Factory to use at a later stage.
The result is the design in \Fref{fig:FactoryMethod}.
The client code will mostly be working with the interfaces in white.

\begin{figure}[h]
	\centering
	\includegraphics{FactoryMethod.1}
	\caption{Factory Method design}
	\label{fig:FactoryMethod}
\end{figure}

For problem one, functions can create objects from a \textit{Factory$<$Button$>$} without worrying if it is working with the brand or fancy factory.
The instantiation in \textit{BrandFactory} is the only line needing to swap to a red button for problem two.
The single logic decision point will be the only client code containing the Brand and Fancy factory classes.
No client code will contain the blue or glossy button classes.

\subsubsection{Consequences}
The client code is no longer bound to a concrete button.
It now just works with an abstract button.
Also, the logic to choose a factory appears once in the code.

However, this solution does require a new factory to be created for each button type.
If a new transparent button is to be added, then a new opacity factory will be needed.
Maintenance is not compromised since only the single logic point needs to be updated to introduce the new factory.
The rest of the client code still does not care that it is now working with a transparent button since the white interfaces did not change.

\subsection{Abstract Factory}
During the instantiation of classes, four independent sets of problems might exist.
The Abstract Factory design pattern proposes to be a solution to these four problems. \cite{gamma_94_01}

\subsubsection{Problems}
The first problem is when the instantiation and representation of classes need to be separate from the application code.
Keeping data structures in a standard library and not the application code is an example of the first problem.
It can be argued that using Abstract Factory for this problem might be over-engineering the solution \cite{kerievsky_05_01}.

A second problem is the reverse of the first.
When a designer wants to create a library of objects but only expose their interface and not their implementation.
In a GUI library, only exposing the operations on a button and not the fact that the button is blue or glossy is an example of hiding the implementation.
This should remind us of the Factory Method design just created.

The designer wanting to have a family of related objects to be used together is the third possible problem.
Forcing the glossy button to appear with the glossy scroller is an example of wanting the object families together.

Lastly, wanting to swap a family of products for another family of products is the fourth possible problem.
This is, swapping all the glossy GUI items to the flat blue items for the entire application by changing one line will be nice.
Again, reminding us of the Factory Method design.
This time just for more products.

For this report, we will only focus on problems two to fourth.

\subsubsection{Solution}
Problems two and four requires each product to have an abstract definition - called \textit{Abstract Product}.
Doing so will hide the implementation for problem two - effectively solving problem two.
For problem four, all the application code will operate against the interface for a button, scroller, and any other product.
The client code will never operate against concrete implementations.
Thus, swapping from the glossy to the blue elements will not require any additional code changes at the method calls.

Problems three and four both need to control the instantiation of a family of products.
Therefore, a class dedicated to products creation will be needed.
Problem four needs this class to be abstract to swap one family for another - hence it being called \textit{Abstract Factory}.
Everything presented so far is the same as Factory Method's.
We now want to create more than one product.
\Fref{fig:AbstractFactoryInterface} shows the Factory Method design extended to more than one product.

\begin{figure}[h]
	\centering
	\includegraphics{AbstractFactory.1}
	\caption{Interfaces needed for Abstract Factory}
	\label{fig:AbstractFactoryInterface}
\end{figure}

Problem three does not need \textit{AbstractFactory} to be abstract since it has only one family.
\Fref{fig:AbstractFactoryInterface} shows how the concrete brand GUI family maps to all the abstractions.
The same mapping can be seen for a fancy family.
Since both concrete designs have the same interfaces, swapping the one for the other is non-trivial since client code will again only operate against the white interfaces.

\subsubsection{Consequences}
Thus, the \textit{Abstract Factory} pattern makes it easy to group a family of related products and swap one family for another.
By having client code only work against the abstractions, the \textit{Abstract Factory} pattern also isolated the concrete implementations from the client code.
Again, adding a transparent family requires the creation of a transparent factory and each transparent product.
However, only the single logic line in the client needs to be updated as a maintenance exercise.

But, adding a new abstract product to the family creates a drawback \cite{gamma_94_01}.
Each concrete family will have to add its own concrete form of the product too.
So adding a new window product means creating one abstraction and updating the two families.
Thus, the number of classes needing to change is \(1 + n\), where \(n\) is the number of families \cite{bulajic_12_01}.

%% bulajic proposes db & Fowler quote (p1416)

\subsection{Visitor}
Performing an operation on a set of objects can be quite difficult.
The \textit{Visitor} design pattern proposes a solution to three problems \cite{gamma_94_01}.

\subsubsection{Problems}
For the first problem, imagine classes all with different interfaces.
But, an operation needs to be performed against each concrete class.
For example, a button and a scroller have different interfaces.
However, both have to be drawn.
Alternatively, a need might exist to read both aloud for the screen-reader.

Doing unrelated operations with the classes is a second problem.
For a study, a company might want to know the average screen surface area of its GUI elements.
This is unrelated to a GUI library.
Adding surface area methods to GUI classes will pollute the classes.

Lastly, the classes may rarely change as a third problem, but the operations performed on them change often.
Coming back to the study, a week later, finding the most common element color might be needed.
No new elements were added to the GUI library.
Only the need for a new operation exists.

\subsubsection{Solution}
The solution is to look outside the classes.
Thus, creating a new class which knows how to perform only a single operation.
The new class will need to visit each of the classes in the problem space.
This class is called \textit{Visitor} and solves problems one and two.

However, problem three adds a new dimension.
Creating a new operation means creating a new visitor type.
Since they are both visiting the same classes, they are both the same in an abstract sense.
It is only their implementations that differ.
Thus, having an \textit{Abstract Visitor} to represent both is needed.

\begin{figure}[h]
	\centering
	\includegraphics{AbstractVisitor.1}
	\caption{Interfaces needed for Abstract Visitor}
	\label{fig:AbstractVisitorInterface}
\end{figure}

\textit{Abstract Visitor} will have a method for each class it needs to visit as seen in \Fref{fig:AbstractVisitorInterface}.
Requiring the client to remember the method corresponding to each class will not be ideal when the classes reach more than 30.
It is also not ideal for generic pieces of code since the method names are not the same.

\begin{figure}[h]
	\centering
	\includegraphics{VisitorAccept.1}
	\caption{Accept on elements to visit}
	\label{fig:VisitorAccept}
\end{figure}

To solve this, each class has a method to \textit{accept} a visitor as seen in \Fref{fig:VisitorAccept}.
This method calls for another abstraction called \textit{Element} with the \textit{accept} method.
In the \textit{accept} method, each class can call the visitor operation corresponding to it.

\subsubsection{Consequences}
New operations (\textit{Visitors}) can easily be added without touching the classes.
Catering for next week's survey means creating a new visitor without touching button or scroller.
Related operations are now also isolated to each visitor.
Thus the classes are not polluted with unrelated methods.
Visitors also store the state information they need rather than passing it to each function as seen in \textit{NameVisitor}.

However, there are two problems.
First, adding a new class means updating all the visitors with a method for it.
Thus, adding the window element will require an update for each visitor to visit it too.
Second, it is assumed that each class exposes enough information through its public interface for visitors to perform their needed operations.
The scroller may not expose its name.
This will leave the name visitor not being able to get the name of scroll elements.
\section{Metaprogramming}
Metaprogramming is a program that writes another program.
A program operates on data; a metaprogram treats a program as its data \cite{savidis_19_01, anggoro_17_01, sheard_01_01}.
Input to a metaprogram will be referred to as meta-code.
This makes a metaprogram like any regular program.
Therefore, a metaprogram can be refactored, abstracted, turned into library helpers, and tested \cite{lilis_15_01}.
Being able to write a program using code opens up many uses.

% TODO: two picture for regular and meta program

% Is a program in itself [lilis]
% allows code reuse at micro and macro level [savidis]

\subsection{Uses}
There are three main uses for metaprogramming: code optimization, code reuse, and analysis.

\paragraph{Code optimization}
A metaprogram can be used to improve the execution speed of the code.
For example, with a Just-In-Time (JIT) compiler, a metaprogram can optimize blocks of code that are called more often than others \cite{hinsen_13_01} or by caching the results of a method's call \cite{seaton_15_01}.
Another example is the use of Domain-Specific Languages (DSL). 
Here a metaprogram has a better understanding of the code and can apply optimizations unknown to the compiler.
Optimizations include knowing a value can never be negative \cite{hinsen_13_01}, simplifying an expression \cite{sheard_01_01}, or offloading to the GPU \cite{videau_18_01}.

\paragraph{Code reuse}
Metaprogramming can also be viewed as a code reuse tool.
Repeated code - like design patterns \cite{lilis_15_01, alexandrescu_01_01} - can be wrapped behind a metaprogram function that will write the reusable code \cite{savidis_19_01, klabnik_2019_01}.
Complex code can also be translated from a simpler language that end-users can understand \cite{hinsen_13_01}.
The generated code can be anything from one-liners to classes \cite{savidis_19_01}.

\paragraph{Analysis}
Reading a program as input is the last use for metaprogramming.
After reading a program, the metaprogram can analyze its control flow, check types on a dynamic language, or build a proof theorem \cite{sheard_01_01}.

\subsection{Dimensions}
Metaprogramming has many dimensions.
This section will briefly discuss these dimensions, as presented by Lilis and Savidis \cite{savidis_19_01}, by focusing on the relationship between the metalanguage and the object language, the model used for metaprogramming, when the metaprogram is executed, the location of the meta-code, and how the final program is represented.

\subsubsection{Relationship to the object language}
Metaprogramming will output code in some language.
The output language is called the object language, while the metaprogram is written in the metalanguage.
These two languages can be may be different.
If they are different, the system is called heterogeneous.
For a heterogeneous system, the metalanguage can extend the object language with extra features or be a completely new language.
When they are the same, it is called a homogenous system \cite{sheard_01_01}.

%% Indistinguishable - recommended by spinellis & lilis

\subsubsection{Model}
Programming comes in different models, such as procedural, functional, and Object-oriented.
The same is true for metaprogramming, which includes the following.

\paragraph{Macro systems}
A macro system takes input and expands it to some output.
The output can be another macro.
Thus expansion continues until no more macros are left.

The input can come in two forms.
First is \textit{lexical macros}, where the input is a stream of tokens.
These tokens can be anything and do not need to adhere to a syntax.
The second form operates on a specific syntax and is called \textit{syntactic macros}.

Parsing the input can be procedural or pattern-based.
Procedural will use an algorithm to generate the output.
Patterns will match an input pattern to its output.

\paragraph{Reflection systems}
% TODO: expand
Reflection is the process an object follows to look at itself and then modify its structure accordingly.
This is commonly done at run-time but can also happen at compile-time.

\paragraph{Metaobject Protocol (MOP)}
Rather than modifying an object's structure, MOPs modify an object's behavior.
This is done by inheriting from a metaclass that can modify its own behavior.
The modification then affects the subclasses \cite{lee_95_01}.
MOPs can also be used to modify a language's behavior \cite{seaton_15_01}.

\paragraph{Aspect-Oriented Programming (AOP)}
Assume one wants to add logging or performance metrics to all functions in a program.
Modifying each function will add a responsibility that does not form part of the function's duties.
There is also the cost in the time it will take to modify each function.
AOP will \textit{weave} the extra responsibility - called \textit{advice} - into each function.

\paragraph{Generative Programming}
This is like a macro system.
The difference being that generative code is clearly not meta-code to be expanded by the macro system.
They also typically represent their data as an Abstract Syntax Tree (AST). 

\paragraph{Multistage Programming}
Generating object code in stages is made possible by multistage programming.
The generations can either be automatic or require manual annotations \cite{sheard_01_01, taha_04_01}.

\subsubsection{Metaprogram execution}
Three options exist for when the metaprogram can be executed.

\paragraph{Before compilation}
% TODO: merge 1st and 2nd sentence
The metaprogram can be executed before compilation.
This offers the option of using any language for metaprogramming.
The metaprogram takes a source file with meta-code and outputs a file without meta-code.

\paragraph{During compilation}
This option runs the metaprogram as part of the compilation.
This requires the compiler to support metaprogramming, or it needs to have a plugin for metaprogramming.

\paragraph{Run-time}
Lastly, the metaprogram might execute at run-time.
This will require the language execution system to support dynamic code generation and execution.

\subsubsection{Meta-code location}
% The meta-code location dimension takes the location of the meta-code to be used as input by the metaprogram into consideration.
The meta-code location dimension takes into consideration the location of the meta-code to be used as input.

\paragraph{External}
The meta-code can be in an external file and will result in a new file with the object code.
This option is used with the before compilation option and generative model.
Alternatively, suppose this option is used with compilation time execution.
In that case, the file needs to be passed to the compiler with a flag.

\paragraph{Embedded}
The meta-code can also be embedded within the program to be transformed.
This means the source file has a mixture of regular code and meta-code.
Embedded code can have three levels of context-awareness.

\begin{itemize}
	\item Completely unaware: The meta-code only relies on the inputs passed to it.
	      The code immediately after the meta-code is not available to the metaprogram.
	\item Local awareness: The meta-code relies not only on inputs but also on the code immediately after the meta-code.
	\item Global awareness: The meta-code relies on inputs and is aware of all code in the file.
\end{itemize}

\subsubsection{Data representation}
The final dimension to consider is the representation used to hold the final code.
Since the final code is the metaprogram's data \cite{bawden_99_01}, it needs to be held in some type.
Many systems use strings, graphs, or an algebraic data structure \cite{sheard_01_01}:

\paragraph{String}
The final program is held in a string.
This option is not desired since building a class may need hundreds of string append operations spanning hundreds of lines.
The object code will be interleaved with the metaprogram making it hard to distinguish between them.
This makes it easy to construct a string that is not syntactically correct.

\paragraph{Graphs}
A graph type will add structure to the program being built.
Furthermore, it makes a better separation between the object code and the metaprogram code.
However, it still does not guarantee that the structure will be syntactically correct.

\paragraph{Algebraic}
Storing the data as an algebraic expression with type encoding or an AST is the only guarantee of a syntactically correct program.
However, building an AST by hand is hard.
Lisp solved this problem by using \textit{quasiquotes} \cite{bawden_99_01}.

\textit{Quasiquotes} is a form of templating/string interpolation.
It allows writing the data as a ``string'' (enclosed in backtick quotes) in the object language's syntactical form.
This \textit{quoted} ``string'' is then transformed into the desired data structure.
Thus, it acts as a shortcut for constructing an AST \cite{lilis_15_01}.
Placeholders are placed in the \textit{quoted} ``string'' to be replaced with variables from the metaprogram context.
These placeholders need to be identifiable.
Thus, the placeholders are preceded by some \textit{unquote} character \cite{bawden_99_01}.\\

Each dimension has an option used by Rust to enable metaprogramming.

\subsection{Metaprogramming in Rust}
Rust has two metaprogramming functionalities built into the language.
The first has been part of the language for some time and is meant for general metaprogramming.
The second, called \textit{Procedural Macros}, is a newer addition added, in late 2018 \footnote{https://blog.rust-lang.org/2018/12/21/Procedural-Macros-in-Rust-2018.html} and is the focus of this report \cite{klabnik_2019_01}.

The metalanguage for \textit{Procedural Macros} in Rust are coded in Rust syntax.
\textit{Procedural Macros} are homogenous.
From the name \textit{Procedural Macros}, it is also clear it follows the macro model, and the parsing is procedural.
The input stream is also a lexical token stream.

Execution happens during compilation.
This means the macros need to be available to the compiler and need to be precompiled.
Therefore, the macros need to be isolated from client code in a library marked for macro use \footnote{https://doc.rust-lang.org/reference/procedural-macros.html}.
The macro invocation is embedded in the client code.
\textit{Procedural macros} have both local awareness or no awareness depending on the flavor used.
Flavors will be discussed in \Fref{sec:procedural-macro-flavors}.
The data representation is the same as the input - a lexical token stream.

The input lexical token stream does not need to follow a specific syntax.
The metaprogram designer is free to choose this syntax.
However, the output stream needs to be valid Rust code.
Rust tries to keep its standard library as slim as possible while offloading features to libraries.
Given the wide range of possible inputs, no standard library helpers exist for working with a token stream.
However, two Rust libraries exist for working with token streams - the input and output of \textit{Procedural Macros}.

The first is \textit{syn}\footnote{https://docs.rs/syn/1.0.31/syn/index.html} for parsing Rust syntax to a syntax tree.
Other parsers can also be built using \textit{syn}.

The second is \textit{quote}\footnote{https://docs.rs/quote/1.0.7/quote/index.html} for generating a token stream from Rust syntax.
It is a macro that uses the quasiquotes concept from Lisp.
Thus, anything in the \textit{quoted} ``string'' is correctly highlighted, formatted, and autocompleted by an editor.

\subsubsection{Procedural Macro flavors}
\label{sec:procedural-macro-flavors}

\newcommand{\functionh}[1]{\highlight[Yellow!100!]{#1}}
\newcommand{\inputh}[1]{\highlight[Blue!40!]{#1}}
\newcommand{\outputh}[1]{\highlight[Green!40!]{#1}}
\newcommand{\contexth}[1]{\highlight[Red!40!]{#1}}

The three flavors of Procedural Macros are function-like, derive, and attribute macros.

\paragraph{Function-like macros}
These are the most straightforward flavor of procedural macros.
They take an input stream and return an output stream - i.e., they are context unaware.
\Fref{lst:function-like-macro} shows a reflective function-like macro that returns its input unaltered as output.

\libraryCodeFromFile[lastline=7]{Metaprogramming/library/src/lib.highlighted.rs}{function-like-macro}{Declaring a function-like macro}

Lines 1 and 2 import the \textit{proc\_macro} library and the \textit{TokenStream} type in the library.
These two lines will be needed for all macros and will not appear in future examples.
The \textit{\#[proc\_macro]} attribute on line 4 marks the function that follows as a function-like macro.
Line 5 shows it taking one \inputh{input} and returning one \outputh{output} - both of type \textit{TokenStream}.
On line 6, the input is returned unaltered.

Client code will have a call as follows to use the macro.

\clientCodeFromFile[firstline=7,lastline=7]{Metaprogramming/client/src/main.highlighted.rs}

This call has the same \functionh{function name} as the macro.
All function macros are invoked using the \textit{!} (exclamation) sign - called the \textit{macro invocation operator} - to distinguish them from regular function calls.
The call will be replaced with the \outputh{output} ``2 + 3, 5''.
Since the output is invalid Rust code, the compiler will give an error on the call line.
Notice how everything inside the parenthesis (\inputh{2 + 3, 5}) will be passed to \inputh{input}.
A \textit{TokenStream} can be thought of as a list of tokens.
There are four possible token types \footnote{https://doc.rust-lang.org/proc\_macro/enum.TokenTree.html}:

\begin{itemize}
	\item An \textit{Ident} to hold an identifier like a variable name or keyword.
	\item A \textit{Punct} to hold a single punctuation mark.
	\item A \textit{Literal} to hold a literal like an integer value, string, or literal character.
	\item A \textit{Group} to hold an inner/nested \textit{TokenStream} surrounded by brackets.
\end{itemize}

2, 3, and 5 in the above macro call will be literal tokens.
The +(plus) sign and ,(comma) will be punctuations in both streams.
Again, parsing and generating the list will be hard.
The next example shows how to use the \textit{syn} and \textit{quote} libraries to make parsing and generating easier.

% NOTE: here
\paragraph{Derive macros}
These macros are used to implement interface methods on objects as seen in \Fref{lst:derive-macro}.

\libraryCodeFromFile[firstline=9,lastline=26]{Metaprogramming/library/src/lib.highlighted.rs}{derive-macro}{A derive macro using \textit{syn} and \textit{quote}}

Line 1 shows the import for the \textit{syn} library to parse the input list of tokens to a syntax tree.
This is follow by the \textit{quote} macro in the \textit{quote} library for generating a token stream.
Derive macros have the \textit{proc\_macro\_derive} attribute followed by the macro \functionh{name} as seen on line 4.
The \inputh{input} on line 5 is the context the macro is called on.
Thus, derive macros have local context-awareness.

This macro is called by annotating a type with the derive attribute.

\clientCodeFromFile[firstline=9,lastline=11]{Metaprogramming/client/src/main.highlighted.rs}

Here, \textit{SomeStruct} is the type being annotated and the \contexth{context} passed to the macro.
Line 6 parses the context to a \textit{DerivedInput} syntax tree from \textit{syn}.
\textit{Syn} will give a compilation error if the parsing fails.
Getting the struct's name happens on line 7.

Lines 9 to 15 show the use of the \textit{quote} library for quasiquotes.
Rather than a \textit{quoted} ``string'', it uses the \textit{quote} macro.
Placeholders are \textit{unquoted} with the \# (pound) sign.
Thus, \textit{\#name} will come from the metaprogram context - line 7.
Notice the syntax highlighting being correct inside the quote macro.
Finally, line 17 converts \textit{output} to a \textit{TokenStream}.

The compiler will append the \outputh{output} below the annotated type for derive macros.
Thus, the \outputh{output} was written by the compiler and not the client.

\paragraph{Attribute macros}
A function-like macro with context-awareness results in attribute macros.
Therefore, two token streams are passed to them.
This time the attribute above the function is \textit{proc\_macro\_attribute} as seen in \Fref{lst:attribute-macro}.

\libraryCodeFromFile[firstline=28]{Metaprogramming/library/src/lib.highlighted.rs}{attribute-macro}{Declaring an attribute macro}

Client code will again annotate a type with an attribute to call the macro.
However, the attribute is the macro's \functionh{function name}.

\clientCodeFromFile[firstline=13,lastline=14]{Metaprogramming/client/src/main.highlighted.rs}

The \inputh{attribute input} is the first stream passed to the function, while the \contexth{context} is the second.
Like function macros, the \contexth{context} will be replaced with the \outputh{output}.

Since the metalanguage is Rust code itself, it is time to learn more about Rust.

% Can be a blunt instrument [spinellis]
% Language designed with meta-programming [spinellis]

%%% (My objectives)
% Should be same as human code [spinellis]
% Source browsing [lilis]
%% Editing support [lilis p761]
% Should be correct - or at least friendly messages [spinellis]
% Debugger integration [lilis]
% Usage should be easy to read [spinellis]

\section{Rust}
Rust is a relatively new language created by Mozilla to be memory safe yet have low level like performance \cite{klabnik_2019_01}.
Traditionally, memory safe languages will make use of a garbage collector which slows performance \cite{hertz_05_01}.
Garbage collector languages include C\# \cite{robinson_04_01}, Java \cite{gosling_96_01}, Python \cite{martelli_06_01}, Golang \cite{tsoukalos_18_01} and Javascript \cite{flanagan_06_01}.
Languages that perform well use manual memory management, which is not memory safe whenever the programmer is not careful.
Dangling pointers \cite{caballero_12_01}, memory leaks \cite{wilson_92_01}, and double freeing \cite{sharp_13_01} in languages like C and C++ are prime examples of manual memory management problems \cite{konrad_18_01}.
Few languages have both memory safety and performance.
However, Rust achieves both by using a less popular model known as ownership.

\subsection{Ownership}
In the ownership model, the compiler uses statical analysis \cite{rasmussen_2019_01} to track which variable owns a piece of heap data -- this does not apply to stack data.
Each data piece can only be owned by one variable at a time.
The owning variable is called the \textit{owner}.
\cite{klabnik_2019_01}

A variable also has a scope.
The scope starts at the variable declaration and ends at the closing curly bracket of the code block containing the variable.
When the owner goes out of scope, Rust returns the memory by calling the \textit{drop} method at the end of the scope.
Ownership is manifested in two forms -- moving and borrowing.
These two forms are explained next.
\cite{klabnik_2019_01}

\subsubsection{Moving}
Moving happens when one variable is assigned to another.
The compiler's analysis moves ownership of the data to the new variable from the initial variable.
The initial variable's access is then invalidated \cite{klabnik_2019_01}.
An analogy example would be to give a book to a friend.
The friend can do anything from annotating to burning the book as they feel fit since the friend is the book's owner.

\embed[firstline=2,lastline=7]{Example of ownership transfer}{Rust/moved}

In \Fref{lst:Rust/moved}, on line 2, a heap data object is created and assigned to variable \textit{s}.
Line 3 assigns \textit{s} to \textit{t}.
However, because \textit{s} is a heap object, the compiler transfers ownership of the data from \textit{s} to \textit{t} and marks \textit{s} as invalid.

When trying to use the data on line 5, via \textit{s}, the compiler, therefore, throws an error saying \textit{s} was moved.
Any reference to \textit{s} after line 3 will always give a compiler error.

Finally, the scope of \textit{t} ends on line 6.
Since the compiler can guarantee \textit{t} is the only variable owning the data, the compiler can free the memory on line 6.

\embed{Function taking ownership}{Rust/fn-move}

Having ownership moving makes excellent memory guarantees within a function; however, it is annoying when calling another function, as seen in \Fref{lst:Rust/fn-move}.
The \textit{take\_ownership} function takes ownership of the heap data resulting in memory cleanup code correctly being inserted at the end of \textit{take\_ownership}'s scope on line 10.
When \textit{main} calls \textit{take\_ownership}, \textit{a} becomes the new owner of \textit{s}'s data, making the call on line 5 invalid.
When taking ownership is not desired, the second form of ownership, borrowing, should be used instead.

\subsubsection{Borrowing}

Borrowing has a new variable take references to data rather than becoming its new owner \cite{klabnik_2019_01}.
An analogy is borrowing a book from a friend with a promise of returning the book to its owner once done with it.

\embed{Function taking borrow}{Rust/fn-borrow}

As seen in \Fref{lst:Rust/fn-borrow}, borrowing makes the function \textit{take\_borrow} take a reference to the data.
References are activated with an ampersand (\textit{\&}) before the type.
Once \textit{take\_borrow} has ended, control goes back to \textit{main} - where the cleanup code will be inserted.
Having references as function parameters is called borrowing \cite{klabnik_2019_01}.
The ampersand is also used in the call argument to signal the called function will borrow the data.

However, in Rust, all variables are immutable by default \cite{klabnik_2019_01}.
Hence changing the data in \textit{take\_borrow} causes an error stating the borrow is not mutable on line 9.
Returning to the borrowed book analogy.
One would not make highlights and notes in a book one borrowed unless the owner gave explicit permission.

\subsection{Immutable by default}
Mutable borrows are an explicit indication that a function/variable is allowed to change the data.

\embed{Function taking mutable borrow}{Rust/fn-mut-borrow}

As seen in \Fref{lst:Rust/fn-mut-borrow}, mutable borrows are activated using \textit{\&mut } on the type.
Again, \textit{mut} is also used in the call to make it explicit the function will modify the data.
Variables -- on the stack or heap -- also need to be declared \textit{mut} to use them as mutable.
\cite{klabnik_2019_01}

The two ownership forms - moving and borrowing - together with mutable variables put some constraints on the code of each variable type and their calls: \cite{klabnik_2019_01}
\begin{itemize}
	\item Moving will always invalidate the variable.
	\item Borrowed variables cannot be mutated.
	      However, more than one function can borrow the data simultaneously in parallel and concurrent code.
	\item Mutable borrowing does allow mutations.
	      But only one function can hold a mutable borrow at a time and no other immutable borrows can exist.
\end{itemize}

The constraints will always be enforced by the compiler, thus requiring all code to meet them.
Meeting these constraints also requires some shift in thinking at times.
Another shift is required because Rust may not classify as an Object-Oriented Programming (OOP) language.

\subsection{Not quite OOP}
No single definition exists to qualify a language as Object-Oriented \cite{meyer_97_01,stefik_85_01,gamma_94_01,klabnik_2019_01}.
Three Object-Oriented definitions will be explored to understand Rust better.
These three are Objects as Data and their Behaviour, Encapsulation, and Inheritance.

\subsubsection{Objects as Data and their Behaviour}
The first definition of Object-Oriented design is a language using objects.
An object, in turn, holds both data and procedures operating on the data \cite{meyer_97_01,stefik_85_01,gamma_94_01}.
Rust meets the data with their operations requirement by holding the data in \textit{struct}s and having the operations defined in \textit{impl} blocks \cite{klabnik_2019_01}.

\paragraph{Struct}
A \textit{struct} is the same as a \textit{struct} in C \cite{stroustrup_13_01} and other C like languages \cite{robinson_04_01, savitch_15_01, malik_09_01}.
Structs are used to define objects with named data pieces, as shown in \Fref{lst:Rust/oop} lines 1 to 5.
Each of the struct properties is named followed by a type.

\embed[lastline=44,pos=p!]{Example of a Struct}{Rust/oop}

\paragraph{Methods}
The operations to perform on a struct are defined in an \textit{impl} block, as seen in lines 7 - 19 in \Fref{lst:Rust/oop}.
Notice how the ownership rules apply to the struct.

The methods \textit{get\_age} and \textit{have\_birthday} will take a borrow of the struct object.
To age, while having a birthday, \textit{have\_birthday} needs to take a mutable borrow of \textit{self} - also why \textit{bar} needs to be \textit{mut} in \textit{main}.
The method \textit{have\_burial} moves \textit{self}, thus invalidating any objects of \textit{Foo} after \textit{have\_burial} is called - resulting in a compile error on line 33.
The same error happened in \Fref{lst:Rust/fn-move}.

\begin{notebox}
	Rust does not always use the \textit{return} keyword.
	The missing semi-colon on line 9 is deliberate and the Rust way to say we want to return the age for the Foo instance.
\end{notebox}

\subsubsection{Encapsulation}
The next definition deals with hiding implementation details from the client - known as encapsulation \cite{klabnik_2019_01, meyer_97_01}.
Encapsulation allows the struct creator to change the internal procedures of the struct without affecting the public interface used by clients.
Rust also meets the encapsulation definition by using the \textit{pub} keyword.

\paragraph{pub}
As can be seen in \Fref{lst:Rust/oop}, the \textit{Foo} struct and its \textit{name} data member is made public explicitly.
The methods \textit{get\_age} and \textit{have\_birthday} are also made public explicitly.
All the other data members and methods are private - Rust's default - and unreachable by client code.
A curious question then is why can \textit{main} access the private variables and methods.
The answer being: \textit{main} is located in the same module/file as the struct and therefore has full access to it.

\begin{notebox}
	Rust does not provide constructors like other OOP languages.
	Instead, Rust has what it calls associate methods.
	An associate method is a method definition not containing \textit{self} in the parameter list \cite{klabnik_2019_01}.
	A constructor like associate-method will return an owned instance of the struct being constructed.
	\Fref{lst:Rust/oop}, lines 36 to 44 shows an example of an associate method for Foo defined in the same file as Foo's definition.
	Yes, Rust code can have multiple \textit{impl} blocks for a single struct.
\end{notebox}

\subsubsection{Inheritance}
The last definition being looked at is inheritance.
Inheritance has an object inherit some of its data members and procedures from a parent object \cite{meyer_97_01, stefik_85_01, gamma_94_01}.
Inheritance is mostly used to reduce code duplication.
Rust does not meet the inheritance definition for OOP.

The Gang of Four made their design patterns for Object-Oriented languages.
They also define two principles for Object-Oriented design \cite{gamma_94_01}.
\begin{itemize}
	\item ``Program to an interface, not an implementation.''
	      Rust meets this requirement by using \textit{traits}.
	\item ``Favor object composition over class inheritance.''
	      The lack of inheritance will, therefore, not be an issue in Rust.
	      Not having inheritance causes Rust code to use composition naturally.
\end{itemize}

Rust \textit{traits} cause it to meet both requirements allowing Rust to implement the Gang of Four design patterns.

\subsection{Traits}
\textit{Traits} are similar \cite{klabnik_2019_01} to \textit{interfaces} in other languages like C\# \cite{robinson_04_01} and Java \cite{gosling_96_01}.
In C++, \textit{abstract classes} are the equivalent of \textit{interfaces} \cite{malik_09_01,stroustrup_13_01,alexandrescu_01_01}.
Thus, \textit{traits} allow the definition of abstract behavior as seen in \Fref{lst:Rust/traits}, lines 1 to 7.

\embed[firstline=3,lastline=35,pos=p!]{Working with traits}{Rust/traits}

The compiler uses \textit{traits} at compile time to guarantee an object implements a set of methods using \textit{trait bounds}.
Line 9 shows the \textit{work} function having a \textit{trait bound} on the generic \textit{T} type.
Thus, any object choosing to implement the \textit{Show} trait can be passed to \textit{work}.
In turn, \textit{work} knows it is safe to call \textit{show()} on the object for type \textit{T}.

More complex trait bounds can be constructed as shown on line 13.
Here \textit{T} (\textit{first}) has to implement the traits: \textit{Show} to be able to call the \textit{show()} method on \textit{first}; \textit{Display} to pass it to \textit{println!}; \textit{PartialEq} to compare it with \textit{U} (\textit{second}).
Trait bounds this complex become hard to read, so Rust offers an alternative syntax for placing trait bounds on generics as seen in lines 19 to 22.

Traits are implemented using the \textit{impl} keyword followed by the trait name as seen on line 29.
The \textit{Show} trait has a default implementation for \textit{show\_size}.
Thus, \textit{Tester} does not need to implement \textit{show\_size} and chooses only to implement the \textit{show} method.

Generics and traits are a good match since they result in static dispatch function calls.

\subsubsection{Static Dispatch}
When Rust sees generics on a function or type, Rust uses what it calls \textit{monomorphization} \cite{klabnik_2019_01}.
\textit{Monomorphization} creates a new function or type at compile time for each concrete object passed into the generic placeholders.
Thus, at compile-time, the compiler knows exactly which version of the expanded function to call.
Knowing which function to call at compile-time is known as \textit{compile-time binding} \cite{malik_09_01} or \textit{static dispatch} \cite{klabnik_2019_01, alexandrescu_01_01}.
Using just \textit{trait objects} - rather than generics - will result in dynamic dispatch.

\subsubsection{Dynamic Dispatch}
Dynamic dispatch \cite{alexandrescu_01_01, klabnik_2019_01} happens when the compiler cannot determine which method to call because the \textit{self} object is not fixed.
Instead, at run-time, pointers inside the trait objects are used to determine which method to call \cite{klabnik_2019_01}, hence why it is also known as \textit{run-time binding} \cite{malik_09_01}.

\Fref{lst:Rust/traits-dyn-dispatch} shows \textit{work} with dynamic dispatch rather than generics.

\embed[firstline=5,lastline=7]{Dynamic Dispatch with \textit{dyn}}{Rust/traits-dyn-dispatch}

Three changes need to be made to the function signature.
\begin{itemize}
	\item The generic \textit{T} is removed and replaced with \textit{Show}.
	\item The \textit{dyn} keyword is added to the type to indicate dynamic dispatch will take place explicitly \cite{klabnik_2019_01}.
	\item Borrowing now takes place.
\end{itemize}

The static dispatch generic trait examples can also be made to take a borrow, but taking ownership gives a compile-time error with dynamic dispatch.
The special \textit{Sized} trait is the cause of the error.

\subsubsection{Sized trait}
Rust keeps all local variables and function arguments on the stack.
Having values on the stack requires their size to be known at compile-time.
A special trait called \textit{Sized} is used by the compiler to make sure this requirement is met. \cite{klabnik_2019_01}

The size of an object is influenced by the data it holds.
Also, any object can choose to implement \textit{Show}.
Ownership will want to pass each object on the stack, but each object will need a different stack size only known at run-time.
However, pointers have a fixed size.
Therefore, putting the dynamic object behind any pointer means the stack size will be fixed.

\begin{notebox}
	\textit{Sized} is a unique Rust trait.
	It is the only trait automatically added as a trait bound on all generics.
	For this reason, a special syntax is needed when the programmer wants to opt-out of the \textit{Sized} trait bound.
	The syntax in question is the \textit{?Sized} trait bound. \cite{klabnik_2019_01}
\end{notebox}

Rust offers many pointer options: \cite{klabnik_2019_01}
\begin{itemize}
	\item A reference - also called a borrow.
	\item The \textit{Box$<$T$>$} used to hold heap objects.
	      \textit{Box} differs from a reference since \textit{Box} owns its \textit{T} \footnote{https://joshleeb.com/blog/rust-traits-trait-objects/}.
	\item The \textit{Rc$<$T$>$} reference counting pointer that is essentially a run-time immutable borrow.
	      When the reference count reaches zero, the \textit{T} is dropped.
	\item The special \textit{RefCell$<$T$>$} which exposes mutable access behind an immutable object using the \textit{Interior Mutability Pattern}.
	\item The combined \textit{Rc$<$RefCell$<$T$>$$>$} - the reference counting pointer allows multiple objects to exist, while the \textit{RefCell} allows mutable access to each existing object.
	\item \textit{Arc$<$T$>$} - the thread-safe version of \textit{Rc}.
	\item \textit{Mutex$<$T$>$} - a thread-safe version of \textit{RefCell}.
	      However, a lock needs to be acquired first.
	\item \textit{RwLock$<$T$>$} - like \textit{Mutex}, but distinguishes between a reader and a writer.
	\item Finally, \textit{Arc$<$RwLock$<$T$>$$>$} which will be used by our Abstract Factory.
	      It will hold \textit{T}s for use in multiple threads using \textit{Arc}, while \textit{RwLock} will guarantee only one writer.
\end{itemize}

One more Rust uniqueness is left to be covered. Rust treats enums differently than other languages.

\subsection{Enums}
In Rust, enums can also hold objects \cite{klabnik_2019_01} as seen in \Fref{lst:Rust/enums}, lines 1 to 3.
The \textit{Option} enum, as defined here, is built into the standard Rust library \cite{klabnik_2019_01} to replace \textit{null} as used in other languages.
An \textit{Option} can either be \textit{Some} object or \textit{None}.
This is yet another design Rust uses to be memory safe \footnote{Tony Hoare, the inventor of the \textit{null} reference, has called the \textit{null} reference a billion-dollar mistake in his 2009 presentation "Null References: The Billion Dollar Mistake" (https://www.infoq.com/presentations/Null-References-The-Billion-Dollar-Mistake-Tony-Hoare/)} by checking the ``\textit{null}'' (\textit{None}) option is handled at compile-time.

\embed{Enums holding objects in Rust}{Rust/enums}

Lines 9 to 12 show the use of \textit{match} - called a \textit{switch} in most languages - to match against each possible enum variant.
Line 10 and 11 are each called a \textit{match arm}.
Line 10 shows how an object can be extracted on an arm and be assigned to a value variable.
If any of the two arms are missing, the compiler will give an error stating not all the enum options are covered.
Because matches need to be exhaustive - i.e. all variants need to be covered - in Rust.
There are two options for getting around the exhaustive check. \cite{klabnik_2019_01}

Either adding the \textit{\_} (underscore) catch-all arm to handle the default case for all missing enum options.
Alternatively, using the \textit{if let} pattern as seen on line 14.
The \textit{if let} pattern also allows extraction of the enum object.
However, since the object is not used inside the if block, no extraction needs to occur.
The \textit{\_} is, therefore, used to not extract the enum object.

Both \textit{if} and \textit{match} blocks are considered expressions in Rust.
Thus, lines 16 and 18 miss their ending semi-colon to return \textit{true} and \textit{false} from the \textit{if}.
The value returned from the \textit{if} is assigned to \textit{valid}.
Returning from a \textit{match} is quite common, especially with a particular enum used for error handling.

\subsubsection{Result enum for error handling}
While other languages use exceptions to propagate errors up to the caller, Rust uses the \textit{Result} enum instead.
The definition for \textit{Result} can be seen in \Fref{lst:Rust/enum-result} on lines 1 to 4.

\embed[lastline=40,pos=p!]{The \textit{Result} enum}{Rust/enum-result}

A function will return \textit{Result} to indicate if it was successful with the \textit{Ok} variant holding the successful value of type \textit{T}.
In the event of an error, the \textit{Err} variant is returned with the error of type \textit{E} - like \textit{may\_error} on line 7.
Any calls to \textit{may\_error} have to handle the possible error.
A few error handling options exist: trying an alternative, panicking, or propagating the error.

\paragraph{Alternative}
Trying an alternative if quite simple.
If the \textit{Err} enum is returned, then just run the alternative instead.

\paragraph{Panicking}
The caller will use a \textit{match pattern} to extract the error and panic as seen on lines 13 to 16.
However, writing matches all the time for possible errors break the flow of the code.
So the \textit{Result} enum has some helper methods defined on it \footnote{https://doc.rust-lang.org/std/result/enum.Result.html} - yes, Rust enums are like ordinary objects that can have methods.

The helper method \textit{expect}, as seen on line 22, does the exact same as the match on lines 13 to 16.

\paragraph{Propagation}
The caller might decide more information is need to panic.
So the caller's caller will need to handle the error instead.

Line 30 shows how to propagate the error up the stack - the \textit{return} is to return from the function and not the match.
Line 29 uses the result if it is fine - the lack of \textit{return} returns from the match and assigns \textit{result} to \textit{r}.
Again, the match is verbose and Rust offers a helper to make it shorter.
The helper is the \textit{?} (question mark) operator.
The \textit{?} operator can be used in any function returning a \textit{Result} or \textit{Option} - or type implementing the \textit{std::ops::Try} trait \cite{klabnik_2019_01}.
Line 37 shows how to use \textit{?} - doing precisely the same as lines 28 to 31.

% macro rules: Hygiene and token list

% Generics
%% Associate types

\section{Reporting}
This section will discuss the implementations for an AF macro and a Visitor macro.
It will first discuss the layout used for the macro library.
Next, hand-written implementation as written by a programmer for AF and Visitor will be presented.
These implementations will be the goalposts for the macro outputs.
Finally, the macros will be written to generate the same output.

\subsection{Library layout}
Parts of the framework being created can be used by other macros/libraries.
Thus, the macro implementations will be separated from the structures they will use.
Another reason for this choice is because the \textit{TokenStream}s, which were presented in \Fref{sec:rust-metaprogramming}, cannot be unit tested.
\textit{Syn} and \textit{quote} thus operate against a wrapper found in \textit{proc\_macro2}\footnote{https://docs.rs/proc-macro2/1.0.19/proc\_macro2/index.html} which the helper structures in this section will also use.
This leads to the libraries shown in \Fref{fig:LibraryLayout}.

\begin{figure}[h]
	\centering
	\includegraphics{Layout.1}
	\caption{Layout of libraries}
	\label{fig:LibraryLayout}
\end{figure}

Client code will use the \textit{macro-patterns} library.
\textit{Macro-patterns} will contain macro definitions as was defined in \Fref{sec:rust-metaprogramming} for Abstract Factory and Visitor.
\textit{Macro-lib} will provide syntax tree components that are missing from \textit{syn} or are simpler than \textit{syn}'s.
Finally, all the code is tested with \textit{macro-test-helpers} providing helpers dedicated to making tests easier.
The tests will not be covered in this report, but the reader should note that automated tests are used to ensure the macro outputs are identical to the hand-written implementations covered next.

\subsection{Hand-written implementations}
\label{sec:hand-written}
Typically the design pattern implementations will be written by a programmer without reusing code.
Even though this section creates macros to replace this manual process, the design patterns will be implemented here manually to know what the macro outputs should be.

\subsubsection{Simple GUI}
The design pattern implementations are built on the simple GUI library shown in \Fref{lst:client-gui} which defines:

\begin{itemize}
	\item An \textit{Element} that can create itself with a given name and return that name.
	\item A \textit{Button} that is an \textit{Element} to be clicked with text.
	\item An \textit{Input} element that can hold text inputs.
	\item A \textit{Child} enum that can be a \textit{Button} or \textit{Input}.
	\item A concrete \textit{Window} struct that can hold \textit{Child} elements.
\end{itemize}

\libraryCodeFromFile[lastline=38]{macro-client/src/gui/elements.rs}{client-gui}{Simple GUI defined by client}

The abstract \textit{Button} and \textit{Input} each have a concrete brand version -- i.e. \textit{BrandButton} and \textit{BrandInput} shown in \Fref{lst:brand-elements}.

\libraryCodeFromFile{macro-client/src/gui/brand_elements.rs}{brand-elements}{Brand GUI elements}

\subsubsection{Hand-written Abstract Factory implementation}
\Fref{lst:abstract-factory-hand} shows an implementation of AF for the GUI.
This implementation maps directly to the UML presented for AF in \Fref{sec:design-abstract-factory}.

\libraryCodeFromFile[firstline=5,lastline=17]{macro-client/src/abstract_factory_hand.rs}{abstract-factory-hand}{Hand-written abstract factory}

Line 2 imports the concrete brand elements from \Fref{lst:brand-elements}, with line 3 importing the abstract elements from \Fref{lst:client-gui}.
A factory method is defined on lines 6 to 8.
On line 10 an AF is defined using the factory method as super traits.
The \textit{Display} super trait is to show the macro can handle complex AFs.

Client code will create a concrete brand factory as follow:

\clientCodeFromFile[firstline=19,lastline=37]{macro-client/src/abstract_factory_hand.rs}

\subsubsection{Hand-written Visitor implementation}
\label{sec:visitor-hand-written}
% TODO: refer back to the UML (here and macro)
A hand-written visitor implementation can be seen in \Fref{lst:visitor-hand}.
Visitor consists of three parts:
\begin{itemize}
	\item The abstract visitor as defined on lines 3 to 14 which maps to the UML for visitor as defined in \Fref{sec:design-visitor}.
	\item Helper functions for traversing the object structure \cite{gamma_94_01} on lines 16 to 37.
	      This allows for default implementations on the abstract visitor to call its respective helper.
	      Doing this allows the client to write less code when their visitor will not visit each element.
	      It means client code does not need to repeat code to visit into an element's children since the client can call a helper that has the traversal code - like \textit{visit\_window} on line 27.
	\item Double dispatch reflections on lines 40 to 58.
	      With these, the client does not need to remember/match each abstract visitor method with the element they are currently using.
	      The client can just call \textit{apply} on the element and have it redirect to the correct visitor method.
\end{itemize}

\libraryCodeFromFile[firstline=5,lastline=62]{macro-client/src/visitor_hand.rs}{visitor-hand}{Hand-written visitor}

A client will write a concrete visitor as shown below.
This visitor collects the names of each element in a structure except for the names of windows to show the power of the default implementations delegating to the helpers.
Because the default implementation in \Fref{lst:visitor-hand} line 12 uses the helper on line 27, \textit{VisitorName} does not need to implement anything for \textit{Window} to be able to travers into a \textit{Window}'s children.
This visitor implements the \textit{Display} trait to be able to call \textit{to\_string()} on it.
Calling \textit{to\_string()} will join all the names this visitor collected.

\clientCodeFromFile[firstline=64,lastline=88]{macro-client/src/visitor_hand.rs}

The next client code shows how to use this visitor.
First a window, button, and input is created.
A random name is set on the input before it and the button is added to the window.
A \textit{VisitorName} is created and applied to the window.
Lastly, a test shows the visitor collected the correct names.

\clientCodeFromFile[firstline=111,lastline=128]{macro-client/src/visitor_hand.rs}

\subsection{Macros}
Rust's metaprogramming abilities will be used to create macros that can write the repeated sections in the hand-written implementations.
The outputs of each macro should be exactly the same as the manually implementations written by a programmer.
Three macros will be created in total: one to create an AF; one to implement a concrete factory for an AF; one to create a Visitor.

\subsubsection{AF macro}
The implementation of the AF macro will be used as a foundation to implement the Visitor macro.
The input passed to the macro -- defined as meta-code in \Fref{sec:metaprogramming} -- will be parsed to a model.
This model will be able to expand itself into its pattern implementation as defined in \Fref{sec:hand-written}.
A model will be composed of syntax elements.
Some of the syntax elements will come from \textit{syn} and others will have to be created.

The client meta-code for AF is as follows -- since it is the same as the hand-written implementation, it also maps directly to the AF UML given in \Fref{sec:design-abstract-factory}:
% TODO: coloring will help

\clientCodeFromFile[firstline=5,lastline=36]{macro-client/src/abstract_factory.rs}

The client needs to specify the factory method they will use.
This factory method needs to take a generic element \textit{T}.
The \textit{AbstractGuiFactory} is annotated with an attribute macro (see \Fref{sec:attribute-macro}) named \textit{abstract\_factory}.
The factory method and factory elements are passed to the macro.

The client will create their concrete \textit{BrandFactory} and use the \textit{interpolate\_traits} attribute macro to implement the factory method for each element.
Here the client uses two invocations of \textit{interpolate\_traits} since \textit{Window} is concrete and does not use the \textit{dyn} keyword.

\paragraph{Models}
Both \textit{abstract\_factory} and \textit{interpolate\_traits} take in a comma-seperated list of inputs.
The \textit{syn} library provides the \textit{Punctuated}\footnote{https://docs.rs/syn/1.0.48/syn/punctuated/struct.Punctuated.html} type to parse a list of elements seperated by any punctuation marker.
\textit{Syn} also provides \textit{Type}\footnote{https://docs.rs/syn/1.0.48/syn/enum.Type.html} for parsing Rust types that will used by the \textit{abstract\_factory} macro.
The elements passed to \textit{interpolate\_traits} need to be custom made.
Two models need to be created for the AF macros:

\begin{enumerate}
	\item A \textit{TraitSpecifier} to hold an item passed to the \textit{interpolate\_traits} macro.
	      Each item will map a trait to its corresponding concrete type.
	\item \textit{AbstractFactoryAttribute} to hold the input passed to the \textit{abstract\_factory} macro.
	      The input will consist of a factory method and a list of elements the AF will create.
\end{enumerate}

\textit{TraitSpecifier} is defined in \Fref{lst:trait-specifier}.
It will use the syntax \textit{trait =$>$ concrete} to map a trait type to its concrete definition.

\libraryCodeFromFile[firstline=5,lastline=29]{macro-lib/src/trait_specifier.rs}{trait-specifier}{Making a parsable trait specifier}

Lines 1 and 2 import the \textit{syn} elements that will be used.
Line 9 is used by the tests.
The model is defined on lines 10 to 14 to hold the abstract trait, the arrow token, and the concrete.
The \textit{Token}\footnote{https://docs.rs/syn/1.0.48/syn/macro.Token.html} macro on line 12 is a helper from \textit{syn} to easily expand Rust tokens and punctuations.
Lines 17 to 25 implement the \textit{Parse}\footnote{https://docs.rs/syn/1.0.48/syn/parse/trait.Parse.html} trait from \textit{syn} for parsing a token stream to this model.
Here parsing is simple, \textit{parse} each stream token or propagate the errors.
\textit{Syn} will take care of converting the errors into compiler errors.

\textit{AbstractFactoryAttribute} is defined in \Fref{lst:abstract-factory-attribute-model}.
This will be the input passed to the \textit{abstract\_factory} macro.

\libraryCodeFromFile[firstline=7,lastline=27]{macro-patterns/src/abstract_factory.rs}{abstract-factory-attribute-model}{Meta-code model for abstract factory macro}

The model takes the factory method trait as the first input, separated (\textit{sep}) by a comma, followed by a comma-separated list of types the abstract factory will create as was shown in the client meta-code.

The expand method for the \textit{AbstractFactoryAttribute} model is defined as seen in \Fref{lst:abstract-factory-attribute-expand}.

\libraryCodeFromFile[firstline=29,lastline=48]{macro-patterns/src/abstract_factory.rs}{abstract-factory-attribute-expand}{Expanding an abstract factory macro model}

The expand method takes in a trait definition syntax tree as \textit{input\_trait} on line 3.
On lines 4 to 11 a factory method super trait is created for each type passed to the macro.
Lines 5 and 6 create local variables to be passed to \textit{quote}.
Line 8 uses a \textit{syn} helper function to turn a \textit{quote} into a syntax tree.
Since \textit{types} defined on line 5 is a list, \textit{quote} has to be told to expand each element in the list.
The special \textit{\#(list-quote)$<$sep$>$*} quasiquote is used to specify how to expand a list.
The optional \textit{sep} character is used as a separator for each item.
On line 9 the factory method is expanded for each type using the + (plus) sign as a separator.
Thus, if \textit{MyFactory, Type1, Type2} is passed to the macro, then line 9 will create \textit{MyFactory$<$Type1$>$ + MyFactory$<$Type2$>$}.

Line 14 appends the factory super trait that was just constructed to the context input.
The new context input is returned on line 17.

\paragraph{Definitions}
The AF macro is shown in \Fref{lst:abstract-factory-macro} to be an attribute macro as was defined in \Fref{sec:attribute-macro}.
Line 11 parses the input context -- which is the \textit{AbstractGuiFactory} definition in the meta-code -- with line 12 parsing the macro inputs to \textit{AbstractFactoryAttribute} as defined in \Fref{lst:abstract-factory-attribute-model}.
Line 14 expands the inputs on the context as defined in \Fref{lst:abstract-factory-attribute-expand}.

\libraryCodeFromFile[firstline=4,lastline=18]{macro-patterns/src/lib.rs}{abstract-factory-macro}{Abstract Factory macro definition}

The \textit{interpolate\_traits} macro -- also being an attribute macro -- is shown in \Fref{lst:interpolate-traits-macro}.

\libraryCodeFromFile[firstline=20,lastline=29]{macro-patterns/src/lib.rs}{interpolate-traits-macro}{Interpolate traits macro definition}

Line 6 parses the macro inputs to a comma-separated list of \textit{TraitSpecifier}s defined in \Fref{lst:trait-specifier}.
Rather than parsing the context input to a model, the context input is used as a template for each concrete factory implementation.
\textit{Quote} macro templates expand when the macro is compiled.
But, these templates need to be expanded when the macro is run.
Macros are run at the compile-time of the client code.
Thus a string interpolator like \textit{quote} is needed that can run at the macro's run-time.
\Fref{lst:fn-interpolate} defines such a helper for a \textit{proc\_macro2} token stream.

\libraryCodeFromFile[lastline=56]{macro-lib/src/token_stream_utils.rs}{fn-interpolate}{Run-time string interpolator}

Line 7 defines an \textit{Interpolate} trait for types that will be interpolatable at macro run-time.
Line 8 implements the \textit{Interpolate} trait for \textit{syn}'s \textit{Punctuated} type if the punctuated tokens implement the \textit{Interpolate} trait -- the \textit{TraitSpecifier} token will be made interpolatable in the next listing.

The \textit{interpolate} function on line 24 takes in a template \textit{stream} and hash map of items to replace in the input stream.
Thus, if the hash map has a key of \textit{TRAIT} with the value of \textit{Window}, then each \textit{TRAIT} in the template will be replaced with \textit{Window}.
Line 28 creates a \textit{new} token stream that will be returned from the function on line 55.
Each token in \textit{stream} will be copied to \textit{new} if the token does not need to be replaced.

Line 30 starts looping through the tokens and line 31 matches on the token type.
Four token types were presented in \Fref{sec:function-like-macro}.
The \textit{Literal}, \textit{Punct}, and \textit{Group} tokens will be copied as is.
Since the \textit{Group} token holds its own token stream, it needs to recursively call \textit{interpolate} on its stream and create a new group from the result -- the span that is copied on line 46 is to preserve the context for compilation errors.
Only the \textit{Ident} tokens are matched against the replacements.
Thus, if the identifier matches any of the replacements on line 35, then the replacement value is copied to the \textit{new} stream on line 36.
Otherwise, the identifier is copied on line 40.

\Fref{lst:trait-specifier-interpolate} shows interpolation being implemented for the \textit{TraitSpecifier} -- specified in \Fref{lst:trait-specifier}.
Lines 5 and 6 set up the hash map to replace \textit{TRAIT} with the abstract trait and \textit{CONCRETE} with the concrete type.
Line 8 calls \textit{interpolate} as defined in \Fref{lst:fn-interpolate} line 24.

\libraryCodeFromFile[firstline=31,lastline=40]{macro-lib/src/trait_specifier.rs}{trait-specifier-interpolate}{Implement \textit{Interpolate} for \textit{TraitSpecifier}}

Thus, line 9 in \Fref{lst:interpolate-traits-macro} will use the context input as a template to interpolate each \textit{TraitSpecifier} passed into the \textit{interpolate\_traits} macro.

\subsubsection{Visitor macro}
The Visitor macro implementation will be a function-like macro -- as was defined in section \Fref{sec:function-like-macro}.
Like the AF implementation, it will use \textit{syn} to parse the input to a model.
The model will be expanded to match a hand-written implementation using \textit{quote}.

The following shows the client meta-code for the Visitor macro -- meta-code being the input to the macro as defined in \Fref{sec:metaprogramming}.
This will result in the same code as \Fref{lst:visitor-hand}.

\clientCodeFromFile[firstline=2,lastline=21]{macro-client/src/visitor.rs}

A list of types is being passed to the \textit{visitor} macro function.
A type can also have two options inside the \textit{\#[options]} syntax:
\begin{enumerate}
	\item \textit{no\_default} to turn off the default trait function implementation -- as defined in \Fref{sec:visitor-hand-written}.
	\item \textit{helper\_tmpl} to modify the helper template used -- also defined in \Fref{sec:visitor-hand-written}.
\end{enumerate}

The client code above shows how the \textit{helper\_tmpl} option is used on the \textit{Window} type.
\textit{Syn} does not make provision for parsing complex options like this.
Thus, this section will create new syntax elements to parse the input for the Visitor macro.

\paragraph{Models}
Six parsable models need to be created:
\begin{enumerate}
	\item \textit{KeyValue} to parse a \textit{key = value} stream.
	      The \textit{key} will be an option with \textit{value} being the option value.
	\item \textit{OptionsAttribute} to hold a comma seperated list of \textit{options} inside the \textit{\#[options]} syntax.
	      Each option will be a \textit{KeyValue}.
	\item \textit{SimpleType} to parse each type in the input list.
	      The \textit{Type} provided by \textit{syn} holds a punctuated \textit{PathSegment}s\footnote{https://docs.rs/syn/1.0.48/syn/struct.PathSegment.html}.
	      The type will determine the function name, thus building a function name from each identifier in the \textit{PathSegment} list is unnecessarily complex.
	\item \textit{AnnotatedType} which is a type annotated with an \textit{OptionsAttribute} like \textit{Window} in the client code above.
	\item \textit{VisitorFunction} to parse the input passed to the macro.
	      The input is a list of \textit{AnnotatedType}s.
\end{enumerate}

The \textit{KeyValue} type is the most complex to parse.
It is defined in \Fref{lst:key-value}.

\libraryCodeFromFile[lastline=49]{macro-lib/src/key_value.rs}{key-value}{Parses a single key value option}

\textit{KeyValue} parses a single \textit{key = value}.
The \textit{key} is an identifier with \textit{value} holding a token stream -- lines 9 to 11.
The \textit{value} part is optional for boolean options.
Thus, line 20 checks if a \textit{value} part is present.
% NOTE: special because of comma used in normal sentences.
A \textit{value} will be present if the end of the stream has not been reached or if the next token is not a comma (,) -- indicating the next key value option.
If no \textit{value} is given, lines 21 to 25 returns the key from the parse function.

Line 29 parses the = (equal) sign with the \textit{value} being parsed on lines 32 to 41.
The \textit{value} needs to be parsed as a token stream, but only a single token needs to be parsed from the current stream.
The \textit{step()} method\footnote{https://docs.rs/syn/1.0.48/syn/parse/struct.ParseBuffer.html\#method.step} with the \textit{token\_tree()} method\footnote{https://docs.rs/syn/1.0.48/syn/buffer/struct.Cursor.html\#method.token\_tree} allows the extraction of a single token leaving the \textit{rest} of the stream intact.
If no \textit{value} was given, line 40 creates a compile error at the current stream position using the \textit{error()} method\footnote{https://docs.rs/syn/1.0.48/syn/parse/struct.StepCursor.html\#method.error}.
Finally, the entire \textit{KeyValue} is returned on lines 43 to 47.

% TODO: can this be token? yes!!!

The \textit{OptionsAttribute} is simple as seen in \Fref{lst:options-attribute}.
It parses the \textit{\#} token, a group of enclosing square brackets, and a comma punctuated list of \textit{KeyValue}s.

\libraryCodeFromFile[lastline=24]{macro-lib/src/options_attribute.rs}{options-attribute}{Parses an attribute with options}

\Fref{lst:simple-type} shows the \textit{SimpleType} model.
It consists of an optional \textit{dyn} keyword followed by a type identifier.
\textit{ToTokens} is implemented on \textit{SimpleType} to be able to use it in \textit{quote} later.

\libraryCodeFromFile[lastline=29]{macro-lib/src/simple_type.rs}{simple-type}{Parses a simple type identifier with an optional \textit{dyn}}

\textit{AnnotatedType} will combine a generic type with an optional \textit{OptionsAttribute} as seen in \Fref{lst:annotated-type}.

\libraryCodeFromFile[lastline=27]{macro-lib/src/rich_type.rs}{annotated-type}{Parses an annotated type}

% TODO: will `Option' like dyn make this simpler

Lastly, \textit{VisitorFunction} -- the input to the Visitor macro -- will be a comma punctuated list of \textit{AnnotatedType}s as shown in \Fref{lst:visitor-function}.
The generic \textit{T} in \textit{AnnotatedType} is set to \textit{SimpleType}.

\libraryCodeFromFile[lastline=131]{macro-patterns/src/visitor.rs}{visitor-function}{Model to parse and expand Visitor macro inputs}

\textit{VisitorFunction} makes use of a private \textit{Options} struct -- lines 94 to 98 -- to dissect the options passed to each type.
On lines 102 to 104 \textit{Options} defaults to creating the default trait method that will call the helper function; creating a helper function; the helper function being empty.

Each option passed to the type is iterated on line 106.
If the option has the \textit{no\_default} key, then creating a default trait method for the type is turned off on line 107 to 110.
The option \textit{helper\_tmpl = false} turns off creating a helper function on 114, while the option \textit{helper\_tmpl = \{template\}} activates a custom helper template on lines 117 to 119.
Line 120 ignores anything else passed to \textit{helper\_tmpl}.

\paragraph{Definition}
\section{Conclusion}
This report showed that Rust's metaprogramming is capable of implementing design patterns that are identical to manual implementations.
The Abstract Factory pattern was first implemented using Rust attribute macros.
The macro definition led to a foundation that can be used to implement other design pattern macros.
The foundation uses a model to parse the macro input into using \textit{syn}.
This model is then expanded to the pattern's implementation using \textit{quote}.
The foundation was applied to create a function-like Rust macro to implement the Visitor pattern.
The same foundation can thus be applied to other design patterns.

Both macros have some limitations that future research can explore.
The Abstract Factory implementation uses dynamic dispatch, which has a performance trade-off.
``Abstract return types''\footnote{https://www.ncameron.org/blog/abstract-return-types-aka--impl-trait-/} may be a solution to this problem.
Visitor's implementation requires the client to supply the traversal code in the \textit{helper\_tmpl} option manually.
However, Rust macros can read the file system.
Thus, it might be possible for the macro the read all the files for a module and build a composition graph of all the module's types.
The traversal code can then be automatically written by the macro -- effectively reducing the visitor macro call to one line.
Lastly, the Visitor has no global option to apply \textit{no\_default} or \textit{helper\_tmpl} to all its types but will require repeating the option on every type.
The macro currently uses an outer attribute\footnotemark on each type to set the option.
An inner attribute\footnotemark[\value{footnote}] can be used on the macro for global options.

The \textit{iterpolate\_traits} macro creates code that almost forwards to a sub-method by mapping one type to another, close to the Decorator design pattern.
Future research can investigate how a new macro will need to change from \textit{interpolate\_traits} to implement Decorator.
The same goes for the Proxy, Adapter, and some parts of Mediator design patterns.

A study of automating this process will also be interesting.
This will be an external metaprogram that identifies repeated patterns in a library, across libraries, or on whole repositories.
It will then create a Rust macro for each repeated pattern and change the identified code sections to use the macro instead.

\footnotetext{https://doc.rust-lang.org/reference/attributes.html}

\bibliographystyle{alpha}
\bibliography{References}

\appendix
\newgeometry{left=1cm,bottom=1.5cm,right=1cm,top=1cm,footskip=1cm}
\section{Appendix}
\subsection{Manual Implementations}
\libraryCodeFromFile{macro-client/src/gui/brand_elements.rs}{brand-elements}
\libraryCodeFromFile{macro-client/src/gui/elements.rs}{client-gui}
\libraryCodeFromFile{macro-client/src/abstract_factory_hand.rs}{abstract-factory-hand}
\libraryCodeFromFile{macro-client/src/visitor_hand.rs}{visitor-hand}

\subsubsection{macro-lib}
\libraryCodeFromFile{macro-lib/src/annotated_type.rs}{annotated-type}
\libraryCodeFromFile{macro-lib/src/key_value.rs}{key-value}
\libraryCodeFromFile{macro-lib/src/options_attribute.rs}{options-attribute}
\libraryCodeFromFile{macro-lib/src/simple_type.rs}{simple-type}
\libraryCodeFromFile{macro-lib/src/token_stream_utils.rs}{fn-interpolate}
\libraryCodeFromFile{macro-lib/src/trait_specifier.rs}{trait-specifier}

\subsubsection{macro-patterns}
\libraryCodeFromFile{macro-patterns/src/abstract_factory.rs}{abstract-factory-attribute}
\libraryCodeFromFile{macro-patterns/src/lib.rs}{macro-patterns}
\libraryCodeFromFile{macro-patterns/src/visitor.rs}{visitor-function}

\end{document}