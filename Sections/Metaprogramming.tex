\section{Metaprogramming}
\label{sec:metaprogramming}
Metaprogramming is using a program that writes another program.
A program operates on data; a metaprogram treats a program as its data, as shown in \Fref{fig:BlackboxProgramming} \cite{savidis_19_01, anggoro_17_01, sheard_01_01}.
Input to a metaprogram will be referred to as meta-code.
This makes a metaprogram like any regular program.
Therefore, a metaprogram can be refactored, abstracted, turned into library helpers, and tested \cite{lilis_15_01}.
Being able to write a program using code opens up many uses.

\begin{figure}[h]
	\centering
	\includegraphics{BlackboxProgramming.1}
	\caption{Programs operating on data}
	\label{fig:BlackboxProgramming}
\end{figure}

% Is a program in itself [lilis]
% allows code reuse at micro and macro level [savidis]

\subsection{Uses}
There are three main uses for metaprogramming: code optimization, code reuse, and analysis.

\paragraph{Code Optimization:}
A metaprogram can be used to improve the execution speed of the code.
For example, with a Just-In-Time (JIT) compiler, a metaprogram can optimize blocks of code that are called more often than others \cite{hinsen_13_01} or by caching the results of a method's call \cite{seaton_15_01}.
Another example is the use of Domain-Specific Languages (DSL). 
Here a metaprogram has a better understanding of the code and can apply optimizations unknown to the compiler.
Optimizations include knowing a value can never be negative \cite{hinsen_13_01}, simplifying an expression \cite{sheard_01_01}, or offloading to the GPU \cite{videau_18_01}.

\paragraph{Code Reuse:}
Metaprogramming can also be viewed as a code reuse tool.
Repeated code -- like design patterns \cite{lilis_15_01, alexandrescu_01_01} -- can be wrapped behind a metaprogram function that will write the reusable code \cite{savidis_19_01, klabnik_2019_01}.
Complex code can also be translated from a simpler language that end-users can understand \cite{hinsen_13_01}.
The generated code can be anything from one-liners to classes \cite{savidis_19_01}.

\paragraph{Analysis:}
Reading a program as input is the last use for metaprogramming.
After reading a program, the metaprogram can analyze its control flow, check types on a dynamic language, or build a proof theorem \cite{sheard_01_01}.

\subsection{Dimensions}
\label{sec:metaprogramming-dimensions}
Metaprogramming has many dimensions.
This section will briefly discuss these dimensions, as presented by Lilis and Savidis \cite{savidis_19_01}, by focusing on the relationship between the metalanguage and the object language, the model used for metaprogramming, when the metaprogram is executed, the location of the meta-code, and how the final program is represented.

\subsubsection{Relationship to the Object Language}
Metaprogramming will output code in some language.
The output language is called the object language, while the metaprogram is written in the metalanguage.
These two languages can be different.
If they are different, the system is called heterogeneous.
For a heterogeneous system, the metalanguage can extend the object language with extra features or be a completely new language.
When they are the same, it is called a homogenous system \cite{sheard_01_01}.

%% Indistinguishable - recommended by spinellis & lilis

\subsubsection{Model}
Programming comes in different models, such as procedural, functional, and Object-Oriented.
The same is true for metaprogramming, which includes the following.

\paragraph{Macro Systems:}
A macro system takes input and expands it to some output.
The output can be another macro.
Thus expansion continues until no more macros are left.

The input can come in two forms.
First is \textit{lexical macros}, where the input is a stream of tokens.
These tokens can be anything and do not need to adhere to a syntax.
The second form operates on a specific syntax and is called \textit{syntactic macros}.

Parsing the input can be procedural or pattern-based.
Procedural will use an algorithm to generate the output.
Patterns will match an input pattern to its output.

\paragraph{Reflection Systems:}
Reflection is the process an object follows to look at its members and methods.
The object can then modify its structure dynamically.
This is commonly done at run-time but can also happen at compile-time.

\paragraph{Metaobject Protocol (MOP):}
Rather than modifying an object's structure, MOPs modify an object's behavior.
This is done by inheriting from a metaclass that can modify its own behavior.
The modification then affects the subclasses \cite{lee_95_01}.
MOPs can also be used to modify a language's behavior \cite{seaton_15_01}.

\paragraph{Aspect-Oriented Programming (AOP):}
Assume one wants to add logging or performance metrics to all functions in a program.
Modifying each function will add a responsibility that does not form part of the function's duties.
There is also the cost in the time it will take to modify each function.
AOP will \textit{weave} the extra responsibility -- called \textit{advice} -- into each function.

\paragraph{Generative Programming:}
This is like a macro system.
The difference being that generative code is clearly not meta-code to be expanded by the macro system.
They also typically represent their data as an Abstract Syntax Tree (AST). 

\paragraph{Multistage Programming:}
Generating object code in stages is made possible by multistage programming.
The generations can either be automatic or require manual annotations \cite{sheard_01_01, taha_04_01}.

\subsubsection{Metaprogram Execution}
Three options exist for when the metaprogram can be executed.

\paragraph{Before Compilation:}
% TODO: merge 1st and 2nd sentence
The metaprogram can be executed before compilation.
This offers the option of using any language for metaprogramming.
The metaprogram takes a source file with meta-code and outputs a file without meta-code.

\paragraph{During Compilation:}
This option runs the metaprogram as part of the compilation.
This requires the compiler to support metaprogramming, or it needs to have a plugin for metaprogramming.

\paragraph{Run-time:}
Lastly, the metaprogram might execute at run-time.
This will require the language execution system to support dynamic code generation and execution.

\subsubsection{Meta-code Location}
% The meta-code location dimension takes the location of the meta-code to be used as input by the metaprogram into consideration.
The meta-code location dimension takes into consideration the location of the meta-code to be used as input.

\paragraph{External:}
The meta-code can be in an external file and will result in a new file with the object code.
This option is used with the before compilation option and generative model.
Alternatively, suppose this option is used with compilation time execution.
In that case, the file might need to be passed to the compiler with a flag.

\paragraph{Embedded:}
The meta-code can also be embedded within the program to be transformed.
This means the source file has a mixture of regular code and meta-code.
Embedded code can have three levels of context-awareness.

\begin{enumerate}
	\item \textbf{Completely unaware}: The meta-code only relies on the inputs passed to it.
	      The code immediately after the meta-code is not available to the metaprogram.
	\item \textbf{Local awareness}: The meta-code relies not only on inputs but also on the code immediately after the meta-code.
	\item \textbf{Global awareness}: The meta-code relies on inputs and is aware of all code in the file.
\end{enumerate}

\subsubsection{Data Representation}
The final dimension to consider is the representation used to hold the final code.
Since the final code is the metaprogram's data \cite{bawden_99_01}, it needs to be held in some type.
Many systems use strings, graphs, or an algebraic data structure \cite{sheard_01_01}:

\paragraph{String:}
The final program is held in a string.
This option is not desired since building a class may need hundreds of string append operations spanning hundreds of lines.
The object code will be interleaved with the metaprogram making it hard to distinguish between them, thus making it easy to construct a string that is not syntactically correct.

\paragraph{Graphs:}
A graph type will add structure to the program being built.
Furthermore, it makes a better separation between the object code and the metaprogram code.
However, it still does not guarantee that the structure will be syntactically correct.

\paragraph{Algebraic:}
Storing the data as an algebraic expression with type encoding or an AST is the only guarantee of a syntactically correct program.
However, building an AST by hand is hard.
Lisp solved this problem by using \textit{quasiquotes} \cite{bawden_99_01}.

\textit{Quasiquotes} is a form of templating/string interpolation.
It allows writing the data as a ``string'' (enclosed in backtick quotes) in the object language's syntactical form.
This \textit{quoted} ``string'' is then transformed into the desired data structure.
Thus, it acts as a shortcut for constructing an AST \cite{lilis_15_01}.
Placeholders are placed in the \textit{quoted} ``string'' to be replaced with variables from the metaprogram context.
These placeholders need to be identifiable.
Thus, the placeholders are preceded by some \textit{unquote} character \cite{bawden_99_01}.\\

Each dimension has an option used by Rust to enable metaprogramming.

\subsection{Metaprogramming in Rust}
\label{sec:rust-metaprogramming}
Rust has two metaprogramming functionalities built into the language.
The first has been part of the language for some time and is meant for general metaprogramming.
The second, called \textit{Procedural Macros}, is a newer addition added in late 2018 \footnote{https://blog.rust-lang.org/2018/12/21/Procedural-Macros-in-Rust-2018.html} and is the focus of this report \cite{klabnik_2019_01}.
\textit{Procedural Macros} use each dimension discussed in \Fref{sec:metaprogramming-dimensions} as follow.

The metalanguage for \textit{Procedural Macros} in Rust is coded in Rust syntax.
Thus, \textit{Procedural Macros} are homogenous.
From the name \textit{Procedural Macros}, it is also clear it follows the macro model, and the parsing is procedural.
The input stream is also a lexical token stream.

Execution happens during compilation.
This means the macros need to be available to the compiler and need to be precompiled.
Therefore, the macros need to be isolated from client code in a library marked for macro use \footnote{https://doc.rust-lang.org/reference/procedural-macros.html}.
The macro invocation is embedded in the client code.
\textit{Procedural macros} have both local awareness or no awareness, depending on the flavor used.
Flavors will be discussed in \Fref{sec:procedural-macro-flavors}.
The data representation is the same as the input -- a lexical token stream.

The input lexical token stream does not need to follow a specific syntax.
The metaprogram designer is free to choose this syntax.
However, the output stream needs to be valid Rust code.
Rust tries to keep its standard library as slim as possible while offloading features to libraries.
Given the wide range of possible inputs, no standard library helpers exist for working with a token stream.
However, two Rust libraries exist for working with token streams -- the input and output of \textit{Procedural Macros}.

The first is \textit{syn}\footnote{https://docs.rs/syn/1.0.31/syn/index.html} for parsing Rust syntax to a syntax tree.
Other parsers can also be built using \textit{syn}.

The second is \textit{quote}\footnote{https://docs.rs/quote/1.0.7/quote/index.html} for generating a token stream from Rust syntax.
It is a macro that uses the quasiquotes concept from Lisp.
Thus, anything in the \textit{quoted} ``string'' is correctly highlighted, formatted, and autocompleted by an editor.

\subsubsection{Procedural Macro Flavors}
\label{sec:procedural-macro-flavors}

The three flavors of Procedural Macros are function-like, derive, and attribute macros.

\paragraph{Function-like Macros:}
\label{sec:function-like-macro}
These are the most straightforward flavor of procedural macros.
They take an input stream and return an output stream -- i.e., they are context unaware.
\Fref{lst:function-like-macro} shows a reflective function-like macro that returns its input unaltered as output.
Note, \Fref{sec:rust} will explain Rust syntax in detail.

\exampleCodeFromFile[lastline=7]{Metaprogramming/library/src/lib.highlighted.rs}{function-like-macro}{Declaring a function-like macro}

Lines 1 and 2 import the \textit{proc\_macro} library and the \textit{TokenStream} type in the library.
These two lines will be needed for all macro definitions and will not appear in future examples.
The \textit{\#[proc\_macro]} attribute in line 4 marks the function that follows as a function-like macro.
Line 5 shows it taking one \inputh{input} and returning one \outputh{output} -- both of type \textit{TokenStream}.
In line 6, the input is returned unaltered.

Client code will have a call as follows to use the macro.

\clientCodeFromFile[firstline=7,lastline=7]{Metaprogramming/client/src/main.highlighted.rs}

This call has the same \functionh{function name} as the macro.
All function macros are invoked using the \textit{!} (exclamation) sign -- called the \textit{macro invocation operator} -- to distinguish them from regular function calls.
The call will be replaced with the \outputh{output} ``2 + 3, 5''.
Since the output is invalid Rust code, the compiler will give an error on the call line.
Notice how everything inside the parenthesis (\inputh{2 + 3, 5}) will be passed to \inputh{input}.
A \textit{TokenStream} can be thought of as a list of tokens.
There are four possible token types \footnote{https://doc.rust-lang.org/proc\_macro/enum.TokenTree.html}:

\begin{itemize}
	\item An \textit{Ident} to hold an identifier like a variable name or keyword.
	\item A \textit{Punct} to hold a single punctuation mark.
	\item A \textit{Literal} to hold a literal like an integer value, string, or literal character.
	\item A \textit{Group} to hold an inner/nested \textit{TokenStream} surrounded by brackets.
\end{itemize}

2, 3, and 5 in the above macro call will be literal tokens.
The +(plus) sign and comma(,) will be punctuations in both streams.
Again, parsing and generating the list will be hard.
The next example shows how to use the \textit{syn} and \textit{quote} libraries to make parsing and generating easier.

\paragraph{Derive Macros:}
These macros are used to automatically implement interface methods on objects, as seen in \Fref{lst:derive-macro}.

\exampleCodeFromFile[firstline=9,lastline=28]{Metaprogramming/library/src/lib.highlighted.rs}{derive-macro}{A derive macro using \textit{syn} and \textit{quote}}

Line 1 shows the import for the \textit{syn} library to parse an unstructured input list of tokens to a syntax tree.
This is followed by the \textit{quote} macro in the \textit{quote} library for generating a token stream.
Derive macros have the \textit{proc\_macro\_derive} attribute with an argument for the macro \functionh{name}, as seen in line 4.
The \contexth{function's input} in line 5 is the context the macro is called on.
Thus, derive macros have local context-awareness.

% TODO: this is unexplained. The next section explains Rust, but this is thin.
% Show how get_type is added to SomeStruct.
This macro is called by annotating a type with the derive attribute.
The type being annotated will serve as the local context for the macro.

\clientCodeFromFile[firstline=9,lastline=10]{Metaprogramming/client/src/main.highlighted.rs}

Here, \textit{SomeStruct} is the type being annotated, and it is the \contexth{context} passed to the macro.
Line 7 parses the context to a \textit{DerivedInput} syntax tree from \textit{syn}.
TokenStreams can be parsed to any syntax tree defined by the macro developer.
Here \textit{DerivedInput} is chosen since it is provided by \textit{syn} for derive macros.
\textit{Syn} will give a compilation error if the parsing fails.
Getting the struct's name happens in line 9.

Lines 11 to 17 show the use of the \textit{quote} library for quasiquotes.
Rather than a \textit{quoted} ``string'', it uses the \textit{quote} macro.
Placeholders are \textit{unquoted} with the \# (pound) sign.
Thus, \textit{\#name} will come from the metaprogram context -- line 9.
Notice the syntax highlighting being correct inside the quote macro.
Finally, line 19 converts \textit{output} to a \textit{TokenStream}.

The compiler will append the macro's \outputh{output} below the annotated type for derive macros.
A snippet of the output is as follows -- it comes from line 12.

\clientCodeFromFile[firstline=11,lastline=11]{Metaprogramming/client/src/main.highlighted.rs}

\paragraph{Attribute Macros:}
\label{sec:attribute-macro}
A function-like macro with context-awareness results in attribute macros.
Therefore, two token streams -- the input and the context -- are passed as arguments to it.
This time the attribute above the function is \textit{proc\_macro\_attribute}.
\Fref{lst:attribute-macro} shows another reflect macro that returns the \contexth{context} unaltered.

\exampleCodeFromFile[firstline=30]{Metaprogramming/library/src/lib.highlighted.rs}{attribute-macro}{Declaring an attribute macro}

Client code will again annotate a type with an attribute to call the macro.
However, the attribute is the macro's \functionh{function name}.

\clientCodeFromFile[firstline=13,lastline=14]{Metaprogramming/client/src/main.highlighted.rs}

The \inputh{attribute's input} -- not following a specific syntax -- is the first stream passed to the function, while the \contexth{context} is the second.
Like function macros, the \contexth{context} will be replaced with the \outputh{output}.

Since the metalanguage is Rust code itself, it is time to learn more about Rust.

% Can be a blunt instrument [spinellis]
% Language designed with meta-programming [spinellis]

%%% (My objectives)
% Should be same as human code [spinellis]
% Source browsing [lilis]
%% Editing support [lilis p761]
% Should be correct - or at least friendly messages [spinellis]
% Debugger integration [lilis]
% Usage should be easy to read [spinellis]
