\section{Rust}
\label{sec:rust}
\newcommand{\errorh}[1]{\textcolor{Red!60!Maroon}{\footnotesize{- #1}}}

Rust is a relatively new language sponsored by Mozilla Research to be memory safe yet have low level like performance \cite{klabnik_2019_01}.
Traditionally, memory safe languages will make use of a garbage collector which slows performance \cite{hertz_05_01}.
Garbage collector languages include C\# \cite{robinson_04_01}, Java \cite{gosling_96_01}, Python \cite{martelli_06_01}, Golang \cite{tsoukalos_18_01} and Javascript \cite{flanagan_06_01}.
Languages that perform well use manual memory management, which is not memory safe whenever the programmer is not careful.
Dangling pointers \cite{caballero_12_01}, memory leaks \cite{wilson_92_01}, and double freeing \cite{sharp_13_01} in languages like C and C++ are prime examples of manual memory management problems \cite{konrad_18_01}.
Few languages have both memory safety and performance.
However, Rust achieves both by using a less popular model known as ownership \cite{matsakis_14_01}.

\subsection{Ownership}
In the ownership model, the compiler uses statical analysis \cite{rasmussen_2019_01} to track which variable owns a piece of heap data -- this does not apply to stack data.
Each data piece can only be owned by one variable at a time, called the \textit{owner} \cite{klabnik_2019_01}.

A variable also has scope.
The scope starts at the variable declaration and ends at the closing curly bracket of the code block containing the variable.
When the owner goes out of scope, Rust returns the memory by calling the \textit{drop} method at the end of the scope.
Ownership is manifested in two forms -- moving and borrowing.
These two forms are explained next \cite{klabnik_2019_01}.

\subsubsection{Moving}
Moving happens when one variable is assigned to another.
The compiler's analysis moves ownership of the data to the new variable from the initial variable.
The initial variable's access is then invalidated \cite{klabnik_2019_01}.
An analogy example would be to give a book to a friend.
The friend can do anything from annotating to burning the book as they feel fit since the friend is the book's owner.

\exampleCodeFromFile{moved/src/main.highlighted.rs}{moved}{Example of ownership transfer}

In \Fref{lst:moved}, in line 2, a heap data object is created and assigned to variable \textit{s}.
Line 3 assigns \textit{s} to \textit{t}.
However, because \textit{s} is a heap object, the compiler transfers ownership of the data from \textit{s} to \textit{t} and marks \textit{s} as invalid.

When trying to use the data in line 4, via \textit{s}, the compiler throws an error saying \textit{s} was moved.
Any reference to \textit{s} after line 3 will always give a compiler error.

Finally, the scope of \textit{t} ends in line 5.
Since the compiler can guarantee \textit{t} is the only variable owning the data, the compiler can free the memory in line 5.

\exampleCodeFromFile{fn-move/src/main.highlighted.rs}{fn-move}{Function taking ownership}

Having ownership moving makes excellent memory guarantees within a function; however, it is annoying when calling another function, as seen in \Fref{lst:fn-move}.
The \textit{take\_ownership} function takes ownership of the heap data resulting in memory cleanup code correctly being inserted at the end of \textit{take\_ownership}'s scope in line 8.
When \textit{main} calls \textit{take\_ownership}, \textit{a} becomes the new owner of \textit{s}'s data, making the call in line 4 invalid.
When taking ownership is not desired, the second form of ownership, borrowing, should be used instead.

\subsubsection{Borrowing}

Borrowing has a new variable take a reference to data rather than becoming its new owner \cite{klabnik_2019_01}.
An analogy is borrowing a book from a friend with a promise of returning the book to its owner once done with it.

\exampleCodeFromFile{fn-borrow/src/main.highlighted.rs}{fn-borrow}{Function taking borrow}

As seen in \Fref{lst:fn-borrow}, borrowing makes the function \textit{take\_borrow} take a reference to the data.
References are activated with an ampersand (\textit{\&}) before the type.
Once \textit{take\_borrow} has ended, control goes back to \textit{main} -- where the cleanup code will be inserted.
Having references as function parameters is called borrowing \cite{klabnik_2019_01}.
The ampersand is also used in the call argument in line 3 to signal the called function will borrow the data.

However, in Rust, all variables are immutable by default \cite{klabnik_2019_01}.
Hence changing the data in \textit{take\_borrow} causes an error stating the borrow is not mutable in line 7.
Returning to the borrowed book analogy.
One would not make highlights and notes in a book one borrowed unless the owner gave explicit permission.

\subsection{Immutable by Default}
Mutable borrows are an explicit indication that a function/variable is allowed to change the data.

\exampleCodeFromFile{fn-mut-borrow/src/main.rs}{fn-mut-borrow}{Function taking mutable borrow}

As seen in \Fref{lst:fn-mut-borrow}, line 6, mutable borrows are activated using \textit{\&mut } on the type.
Again, \textit{mut} is also used in the call in line 3 to make it explicit that the function will modify the data.
Variables -- on the stack or heap -- also need to be declared \textit{mut} to use them as mutable \cite{klabnik_2019_01}, as seen in line 2.

The two ownership forms -- moving and borrowing -- together with mutable variables put some constraints on the code for variables and their calls \cite{klabnik_2019_01}.
The compliler wll always enforce these constraints, requiring all code -- written manually by hand or a macro -- to meet them.
Meeting these constraints also requires some shift in thinking.
Another shift is required because Rust may not classify as an Object-Oriented Programming (OOP) language, which will now be discussed.

\subsection{Not Quite OOP}
No single definition exists to qualify a language as Object-Oriented \cite{meyer_97_01,stefik_85_01,gamma_94_01,klabnik_2019_01}.
Three Object-Oriented concepts will be explored to understand Rust better.
These three are \textit{Objects as Data and their Behaviour}, \textit{Encapsulation}, and \textit{Inheritance}.

\subsubsection{Objects as Data and Their Behaviour}
An object holds both data and procedures operating on the data \cite{meyer_97_01,stefik_85_01,gamma_94_01,malik_09_01}.
Rust holds data in \textit{struct}s with the operations being defined in \textit{impl} blocks \cite{klabnik_2019_01}.

\paragraph{Struct:}
A \textit{struct} is the same as a \textit{struct} in C \cite{stroustrup_13_01} and other C-like languages \cite{robinson_04_01,savitch_15_01,malik_09_01}.
Structs are used to define objects with named data pieces followed by a type, as shown in \Fref{lst:oop} lines 1 to 5.

\paragraph{Methods:}
The operations to perform on a struct are defined in \textit{impl} blocks, as seen in lines 6 -- 16 in \Fref{lst:oop}.

The ownership rules apply to the struct as follow:
\begin{itemize}
	\item The method \textit{have\_burial} moves \textit{self} and will result in cleanup code being inserted in line 15.
	      Thus, any objects of \textit{Foo} after \textit{have\_burial} is called will be invalidated.
	      This happens in line 22, resulting in a compile error -- the same error happened in \Fref{lst:fn-move}.
	\item The method \textit{get\_age} will take a borrow of the struct object.
	\item To age, while having a birthday, \textit{have\_birthday} needs to take a mutable borrow of \textit{self} -- also why \textit{bar} needs to be \textit{mut} in \textit{main} in line 19.
\end{itemize}

\exampleCodeFromFile[lastline=31]{oop/src/main.highlighted.rs}{oop}{Example of a Struct and Methods}

\addtocounter{footnote}{1}\footnotetext{%
	\label{return}
	Rust does not always use the \textit{return} keyword.
	Lines without a semi-colon and that are the last line in a scope act as return statements.
	Thus, line 9 acts as a return statement.
}
\addtocounter{footnote}{1}\footnotetext{\label{pub}\textit{main} can access the private members because main in the same file as them.}
\addtocounter{footnote}{1}\footnotetext{\label{impl}Rust code can have multiple \textit{impl} blocks for a single type.}
\addtocounter{footnote}{1}\footnotetext{%
	\label{associate}
	Rust does not provide constructors like other OOP languages.
	Instead, Rust has what it calls \textit{associate functions}.
	An associate function is a method definition not containing \textit{self} in the parameter list \cite{klabnik_2019_01}.
	They are used as constructors by returning an owned instance of the struct, as seen in line 28.
}

\subsubsection{Encapsulation}
Encapsulation hides the implementation details from the client \cite{klabnik_2019_01,meyer_97_01,savitch_15_01}.
In Rust, encapsulation is the default unless specified otherwise using the \textit{pub} keyword, as seen on lines 2, 7, and 10.
Encapsulation allows the struct creator to change the internal procedures of the struct without affecting the public interface used by clients.
Therefore, Rust meets the encapsulation concept.

\subsubsection{Inheritance}
Inheritance allows an object to inherit some of its data members and procedures from a parent object \cite{meyer_97_01,stefik_85_01,gamma_94_01,savitch_15_01}.
Inheritance is mostly used to reduce code duplication.
Rust does not make provision for the inheritance concept.

Rust does make provision for ``Program to an interface, not an implementation'' and ``Favor object composition over class inheritance'', which the GoF includes as concepts for OOP design \cite{gamma_94_01}.
Both concepts are realized using \textit{traits}, thereby allowing Rust to implement the GoF design patterns.

\textit{Traits} are similar \cite{klabnik_2019_01} to \textit{interfaces} in other languages like C\# \cite{robinson_04_01} and Java \cite{gosling_96_01}.
In C++, \textit{abstract classes} are the equivalent of \textit{interfaces} \cite{malik_09_01,stroustrup_13_01,alexandrescu_01_01}.
Thus, \textit{traits} allow the definition of abstract behavior to program to an interface, as seen in \Fref{lst:traits}, lines 1 to 7.

\exampleCodeFromFile[lastline=18]{traits/src/main.rs}{traits}{Working with traits}

The compiler uses \textit{traits} at compile time to guarantee an object implements a set of methods using \textit{trait bounds}.
Line 9 shows the \textit{work} function having a \textit{trait bound} on the generic \textit{T} type.
Thus, any object choosing to implement the \textit{Show} trait can be passed to \textit{work}.
In turn, \textit{work} knows it is safe to call \textit{show()} on the object for type \textit{T}.

% TODO: use colors to make easier

Traits are implemented on structs using the \textit{impl} keyword followed by the trait name, as seen in line 14.
The \textit{Show} trait has a default implementation for \textit{show\_size} in line 4.
Thus, \textit{Tester} does not need to implement \textit{show\_size}, but is only required to implement the \textit{show} method.

Traits allow one to use composition to construct complex objects.
Getting back to the abstract confirmation dialog presented in \Fref{sec:factory-method-problems} for an open dialog and a safe dialog, using composition, the generic confirmation dialog will have to be composed of a button and text, as seen in \Fref{lst:composition}.

\exampleCodeFromFile{composition/src/lib.rs}{composition}{Using composition}

Using traits, an abstract button and display text is defined in lines 1 to 4.
The confirmation dialog ``has-a'' \cite{malik_09_01} button and display text in lines 7 to 8.
Using associate functions\footnotemark[\ref{associate}], a composition constructor is created in line 12.
To construct the open dialog, the open button will be passed to this constructor.
The safe dialog will need the safe button to be passed in.
Finally, the \textit{show} method in line 15 will show the confirmation dialog, whether it is the open dialog, save dialog, or a new dialog definition.
This is called polymorphism since a dialog will change form depending on the elements passed to the constructor at run-time \cite{savitch_15_01,malik_09_01,gamma_94_01}.

The \textit{show} method will now work with more than one button.
The open button will have a different implementation for the \textit{draw} method, defined in line 2, than the save button.
Since the buttons change at run-time, the compiler will be unable to determine the correct \textit{draw} method to call in \textit{show} in line 16.
Dispatching, as discussed next, resolves this problem \cite{pierce_02_01}.

\subsection{Dispatch}
Dispatching is the matching of the correct method to an object at method invocations \cite{driesen_95_01}.
Sometimes there is only one object to match against, making matching easy.
However, it can also be the case that multiple objects have the same method name, making matching harder.
The matching can happen at compile-time -- called \textit{static dispatching} -- or run-time -- called \textit{dynamic dispatching}.

\subsubsection{Static Dispatch}
Matching the method call at compile-time is called \textit{static dispatch} \cite{klabnik_2019_01, alexandrescu_01_01, abadi_12_01}.
When Rust compiles generics on a function or type, Rust uses what it calls \textit{monomorphization} \cite{klabnik_2019_01}.
\textit{Monomorphization} creates a new function or type at compile time for each concrete object passed into the generic placeholders.

In \Fref{lst:traits}, line 9, the generic \textit{work} method is defined.
Suppose the \textit{work} method is called with a \textit{Tester}, defined in line 13, and later with a \textit{String}.
Then an example of the \textit{monomorphization} process during compilations is shown in \Fref{lst:monomorphization}.

\exampleCodeFromFile[firstline=20,lastline=25]{traits/src/main.rs}{monomorphization}{Example of \textit{monomorphization}}

Each \textit{show} call in the expanded functions can match only one object -- \textit{Tester} and \textit{String} respectively -- making this a simple case.
Using \textit{trait objects} -- rather than generics -- will result in multiple objects having the same method name and the need for dynamic dispatch.

\subsubsection{Dynamic Dispatch}
When the compiler cannot determine which method to call at compile-time, because the object type is not fixed, dynamic dispatch \cite{alexandrescu_01_01, klabnik_2019_01, abadi_12_01} occurs.
At run-time, pointers held by the trait objects are used to determine which method to call \cite{klabnik_2019_01} based on the object type the method is called on.

\Fref{lst:traits-dyn-dispatch} shows the \textit{work} function implemented using dynamic dispatch rather than generics -- refer to \textit{work} in \Fref{lst:traits} line 9 for the generic version.
Since generics are not used, \textit{monomorphization} will not happen.
Calling this function with a \textit{Tester} object means the \textit{show} for \textit{Tester} needs to match at the method invocation in line 2.
But later, calling this function with a \textit{String} object means \textit{String}'s \textit{show} needs to match.
Since the objects passed in changes at run-time, knowing which object to match against can, therefore, only be known at run-time.
Thus, pointers inside the \textit{Show} trait object -- line 1 of \Fref{lst:traits} -- are used at run-time to match each object against its method.

\exampleCodeFromFile[firstline=5,lastline=7]{traits-dyn-dispatch/src/main.rs}{traits-dyn-dispatch}{Dynamic Dispatch with \textit{dyn}}

Three changes need to be made to the function signature to use trait objects rather than generics -- compare to \textit{work} in \Fref{lst:traits}, line 9.
\begin{enumerate}
	\item The generic \textit{T} is removed and replaced with \textit{Show}.
	\item The \textit{dyn} keyword is added to the type to indicate that dynamic dispatch will take place explicitly \cite{klabnik_2019_01}.
	\item Borrowing (\& before \textit{dyn}) now takes place.
\end{enumerate}

The static dispatch generic trait examples in \Fref{lst:traits} can also be made to take a borrow, but taking ownership gives a compile-time error with dynamic dispatch caused by the \textit{Sized} trait.

\subsubsection{Sized Trait}
Rust keeps all local variables and function arguments on the stack.
Having values on the stack requires their size to be known at compile-time.
A special trait called \textit{Sized} is used by the compiler to mark that the size of a type is known at compile-time \cite{klabnik_2019_01}.
This marking is the only use of the \textit{Sized} trait, and it has no meaning or representation after compilation.
However, Rust automatically/implicitly adds the \textit{Sized} trait bound to all function arguments and local variables.

The size of an object is influenced by the data it holds.
Also, any object can choose to implement \textit{Show}.
Thus, two different objects implementing \textit{Show} can have different sizes.
Ownership will want to pass each object on the stack, but with dynamic dispatch, each object will need a different stack size only known at run-time.
Therefore, the \textit{Show} dynamic trait's size cannot be determined at compile-time, thus the \textit{Show} dynamic trait cannot implement the \textit{Sized} trait.

Since all function arguments expect the type to implement the \textit{Sized} trait but having the \textit{Show} dynamic trait not implement it, a compile error will be given stating \textit{dyn Show} does not implement \textit{Sized} when trying to use it as a function argument.
However, pointers do implement \textit{Sized}.
Therefore, putting the dynamic trait behind any pointer will allow it to be used as a function argument or local variable.
The reference (taking a borrow) and the \textit{Box$<$T$>$} type are two such pointers\cite{klabnik_2019_01}.

One more Rust uniqueness is left to be covered. Rust treats enums differently than other languages.

\subsection{Enums}
In Rust, enums can also hold objects \cite{klabnik_2019_01} as seen in \Fref{lst:enums}, lines 1 to 4.
The \textit{Option} enum, as defined here, is built into the standard Rust library \cite{klabnik_2019_01} to replace \textit{null} as used in most languages.
An \textit{Option} can either be \textit{Some} object or \textit{None}.
This is a design Rust uses to be memory safe \footnote{Tony Hoare, the inventor of the \textit{null} reference, has called the \textit{null} reference a billion-dollar mistake in his 2009 presentation "Null References: The Billion Dollar Mistake" (https://www.infoq.com/presentations/Null-References-The-Billion-Dollar-Mistake-Tony-Hoare/)} by handling the (\textit{None}) option at compile-time.

\exampleCodeFromFile{enums/src/main.rs}{enums}{Enums holding objects in Rust}

Lines 9 to 12 show the use of \textit{match} -- called a \textit{switch} in most languages -- to match against each possible enum variant.
Line 10 and 11 are each called a \textit{match arm}.
Line 10 shows how an object can be extracted on an arm and be assigned to a \textit{value} variable.
If any of the arms are missing, the compiler will provide an error stating not all the enum options are covered.
Matches need to be exhaustive since all variants need to be covered in Rust.
The exhaustive check is not always desired.
Thus, there are two options for mitigating the exhaustive check \cite{klabnik_2019_01}.

\begin{itemize}
	\item Adding the \textit{\_} (underscore) catch-all arm to handle the default case for all missing enum options.
	\item Using the \textit{if let} pattern as seen in line 14.
\end{itemize}

The \textit{if let} pattern also allows extraction of the enum object.
Here the extraction will not be used as indicated by the \textit{\_} (underscore).

\textit{Match} blocks and \textit{if} conditions -- which include \textit{if let} -- are considered expressions in Rust.
Thus, lines 16 and 18 are not including their ending semi-colon to return \textit{true} and \textit{false} from the \textit{if} expression which is assigned to the \textit{valid} variable.

\subsubsection{Result Enum for Error Handling}
\label{sec:result-enum}
While most languages use exceptions to propagate errors back to the caller, Rust uses the \textit{Result} enum instead.
The definition for \textit{Result} can be seen in \Fref{lst:enum-result} in lines 1 to 4.

\exampleCodeFromFile[lastline=33]{enum-result/src/main.rs}{enum-result}{The \textit{Result} enum}

A function will return \textit{Result} to indicate if it was successful with the \textit{Ok} variant holding the successful value of type \textit{T}.
In the event of an error, the \textit{Err} variant is returned with the error of type \textit{E} -- like \textit{may\_error} in line 7.
Any calls to \textit{may\_error} have to handle the possible error by panicking or propogating the error.

\paragraph{Panicking:}
The caller will use a \textit{match pattern} to extract the error and panic, as seen in lines 12 to 15.
However, writing matches all the time for possible errors breaks the flow of the code.
So the \textit{Result} enum has some helper methods defined on it \footnote{https://doc.rust-lang.org/std/result/enum.Result.html}.
The helper method \textit{expect}, as seen in line 18, is identical to lines 12 to 15.

\paragraph{Propagation:}
The caller might decide more information is needed to panic.
So the caller's caller will need to handle the error instead.

Line 24 shows how to propagate the error up the stack -- the \textit{return} is to return from the function \textit{error\_explicit\_propogation} and not the match.
Line 29 uses the \textit{result} if it is fine -- the lack of \textit{return} returns from the match and assigns \textit{result} to \textit{r}.
Line 30 shows how to use the \textit{?} (question mark) operator to do the same thing.
The \textit{?} can be used in functions that return \textit{Result}.

% macro rules: Hygiene and token list

% Generics
%% Associate types
